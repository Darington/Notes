\PassOptionsToPackage{unicode=true}{hyperref} % options for packages loaded elsewhere
\PassOptionsToPackage{hyphens}{url}
%
\documentclass[]{article}
\usepackage{lmodern}
\usepackage{amssymb,amsmath}
\usepackage{ifxetex,ifluatex}
\usepackage{fixltx2e} % provides \textsubscript
\ifnum 0\ifxetex 1\fi\ifluatex 1\fi=0 % if pdftex
  \usepackage[T1]{fontenc}
  \usepackage[utf8]{inputenc}
  \usepackage{textcomp} % provides euro and other symbols
\else % if luatex or xelatex
  \usepackage{unicode-math}
  \defaultfontfeatures{Ligatures=TeX,Scale=MatchLowercase}
\fi
% use upquote if available, for straight quotes in verbatim environments
\IfFileExists{upquote.sty}{\usepackage{upquote}}{}
% use microtype if available
\IfFileExists{microtype.sty}{%
\usepackage[]{microtype}
\UseMicrotypeSet[protrusion]{basicmath} % disable protrusion for tt fonts
}{}
\IfFileExists{parskip.sty}{%
\usepackage{parskip}
}{% else
\setlength{\parindent}{0pt}
\setlength{\parskip}{6pt plus 2pt minus 1pt}
}
\usepackage{hyperref}
\hypersetup{
            pdftitle={H2ss},
            pdfauthor={gc5k},
            pdfborder={0 0 0},
            breaklinks=true}
\urlstyle{same}  % don't use monospace font for urls
\usepackage[margin=1in]{geometry}
\usepackage{color}
\usepackage{fancyvrb}
\newcommand{\VerbBar}{|}
\newcommand{\VERB}{\Verb[commandchars=\\\{\}]}
\DefineVerbatimEnvironment{Highlighting}{Verbatim}{commandchars=\\\{\}}
% Add ',fontsize=\small' for more characters per line
\usepackage{framed}
\definecolor{shadecolor}{RGB}{248,248,248}
\newenvironment{Shaded}{\begin{snugshade}}{\end{snugshade}}
\newcommand{\AlertTok}[1]{\textcolor[rgb]{0.94,0.16,0.16}{#1}}
\newcommand{\AnnotationTok}[1]{\textcolor[rgb]{0.56,0.35,0.01}{\textbf{\textit{#1}}}}
\newcommand{\AttributeTok}[1]{\textcolor[rgb]{0.77,0.63,0.00}{#1}}
\newcommand{\BaseNTok}[1]{\textcolor[rgb]{0.00,0.00,0.81}{#1}}
\newcommand{\BuiltInTok}[1]{#1}
\newcommand{\CharTok}[1]{\textcolor[rgb]{0.31,0.60,0.02}{#1}}
\newcommand{\CommentTok}[1]{\textcolor[rgb]{0.56,0.35,0.01}{\textit{#1}}}
\newcommand{\CommentVarTok}[1]{\textcolor[rgb]{0.56,0.35,0.01}{\textbf{\textit{#1}}}}
\newcommand{\ConstantTok}[1]{\textcolor[rgb]{0.00,0.00,0.00}{#1}}
\newcommand{\ControlFlowTok}[1]{\textcolor[rgb]{0.13,0.29,0.53}{\textbf{#1}}}
\newcommand{\DataTypeTok}[1]{\textcolor[rgb]{0.13,0.29,0.53}{#1}}
\newcommand{\DecValTok}[1]{\textcolor[rgb]{0.00,0.00,0.81}{#1}}
\newcommand{\DocumentationTok}[1]{\textcolor[rgb]{0.56,0.35,0.01}{\textbf{\textit{#1}}}}
\newcommand{\ErrorTok}[1]{\textcolor[rgb]{0.64,0.00,0.00}{\textbf{#1}}}
\newcommand{\ExtensionTok}[1]{#1}
\newcommand{\FloatTok}[1]{\textcolor[rgb]{0.00,0.00,0.81}{#1}}
\newcommand{\FunctionTok}[1]{\textcolor[rgb]{0.00,0.00,0.00}{#1}}
\newcommand{\ImportTok}[1]{#1}
\newcommand{\InformationTok}[1]{\textcolor[rgb]{0.56,0.35,0.01}{\textbf{\textit{#1}}}}
\newcommand{\KeywordTok}[1]{\textcolor[rgb]{0.13,0.29,0.53}{\textbf{#1}}}
\newcommand{\NormalTok}[1]{#1}
\newcommand{\OperatorTok}[1]{\textcolor[rgb]{0.81,0.36,0.00}{\textbf{#1}}}
\newcommand{\OtherTok}[1]{\textcolor[rgb]{0.56,0.35,0.01}{#1}}
\newcommand{\PreprocessorTok}[1]{\textcolor[rgb]{0.56,0.35,0.01}{\textit{#1}}}
\newcommand{\RegionMarkerTok}[1]{#1}
\newcommand{\SpecialCharTok}[1]{\textcolor[rgb]{0.00,0.00,0.00}{#1}}
\newcommand{\SpecialStringTok}[1]{\textcolor[rgb]{0.31,0.60,0.02}{#1}}
\newcommand{\StringTok}[1]{\textcolor[rgb]{0.31,0.60,0.02}{#1}}
\newcommand{\VariableTok}[1]{\textcolor[rgb]{0.00,0.00,0.00}{#1}}
\newcommand{\VerbatimStringTok}[1]{\textcolor[rgb]{0.31,0.60,0.02}{#1}}
\newcommand{\WarningTok}[1]{\textcolor[rgb]{0.56,0.35,0.01}{\textbf{\textit{#1}}}}
\usepackage{longtable,booktabs}
% Fix footnotes in tables (requires footnote package)
\IfFileExists{footnote.sty}{\usepackage{footnote}\makesavenoteenv{longtable}}{}
\usepackage{graphicx,grffile}
\makeatletter
\def\maxwidth{\ifdim\Gin@nat@width>\linewidth\linewidth\else\Gin@nat@width\fi}
\def\maxheight{\ifdim\Gin@nat@height>\textheight\textheight\else\Gin@nat@height\fi}
\makeatother
% Scale images if necessary, so that they will not overflow the page
% margins by default, and it is still possible to overwrite the defaults
% using explicit options in \includegraphics[width, height, ...]{}
\setkeys{Gin}{width=\maxwidth,height=\maxheight,keepaspectratio}
\setlength{\emergencystretch}{3em}  % prevent overfull lines
\providecommand{\tightlist}{%
  \setlength{\itemsep}{0pt}\setlength{\parskip}{0pt}}
\setcounter{secnumdepth}{0}
% Redefines (sub)paragraphs to behave more like sections
\ifx\paragraph\undefined\else
\let\oldparagraph\paragraph
\renewcommand{\paragraph}[1]{\oldparagraph{#1}\mbox{}}
\fi
\ifx\subparagraph\undefined\else
\let\oldsubparagraph\subparagraph
\renewcommand{\subparagraph}[1]{\oldsubparagraph{#1}\mbox{}}
\fi

% set default figure placement to htbp
\makeatletter
\def\fps@figure{htbp}
\makeatother

\usepackage{booktabs}
\usepackage{longtable}
\usepackage{array}
\usepackage{multirow}
\usepackage{wrapfig}
\usepackage{float}
\usepackage{colortbl}
\usepackage{pdflscape}
\usepackage{tabu}
\usepackage{threeparttable}
\usepackage{threeparttablex}
\usepackage[normalem]{ulem}
\usepackage{makecell}
\usepackage{xcolor}

\title{H2ss}
\author{gc5k}
\date{2/27/2020}

\begin{document}
\maketitle

{
\setcounter{tocdepth}{2}
\tableofcontents
}
\hypertarget{toc}{%
\section{TOC}\label{toc}}

\begin{Shaded}
\begin{Highlighting}[]
\KeywordTok{print}\NormalTok{(}\KeywordTok{Sys.time}\NormalTok{())}
\end{Highlighting}
\end{Shaded}

\begin{verbatim}
## [1] "2020-04-10 15:42:22 CST"
\end{verbatim}

\hypertarget{vc-mom}{%
\subsection{1 VC-MoM}\label{vc-mom}}

\hypertarget{additive-model}{%
\subsubsection{Additive model}\label{additive-model}}

\(y|e,\mathbf{\beta}=\mathbf{X}\beta+e\)

\(e|\sigma^2_e~N(0,\sigma^2I_N)\)

\(\beta|\sigma^2_g~N(\frac{\sigma^2_g}{M},I_M)\)

\(y\) is centered.

\[
\begin{align}
cov(y)&=\mathbb{E}(yy^T)-\mathbb{y}\mathbb{y}^T\\
&=\sigma^2_a\mathbf{\Omega}_a+\sigma^2_eI_N
\end{align}
\]

The target to be minimized is

\(Q=argmin_{(\sigma^2_a,\sigma^2_e)}|| yy^T-(\sigma^2_a\mathbf{\Omega}+\sigma^2_eI_N)||_F^2\)

Here the subscript \(_{F}\) is for
\href{https://en.wikipedia.org/wiki/Matrix_norm}{Frobenius norm} of a
matrix \(\mathbf{A}\), \(||\mathbf{A}||_F=\sqrt{tr(\mathbf{AA}^T)}\).

\[
\begin{align}
Q&=argmin_{(\sigma^2_a,\sigma^2_e)}|| yy^T-(\sigma^2_a\mathbf{\Omega}_a+\sigma^2_eI_N)||_F^2\\
&=tr\{[yy^T-(\sigma^2_a\mathbf{\Omega}_a+\sigma^2_eI_N)][yy^T-(\sigma^2_a\mathbf{\Omega}_a+\sigma^2_eI_N)]^T\}
\end{align}
\]

An ordinary least squares (OLS) problem of \(Q\) is as below

\[
\left\{
 \begin{array}{lll}
  \frac{\partial{Q}}{\partial{\sigma^2_a}} & =tr\{\frac{yy^Tyy^T-2(\sigma^2_a\mathbf{\Omega}_a+\sigma^2_eI_N)yy^T+(\sigma^2_a\mathbf{\Omega}_a+\sigma^2_eI_N)^2}{\partial{\sigma^2_a}}\}&=tr\{\sigma^2_a\mathbf{\Omega}_a\mathbf{\Omega}_a^T+\sigma^2_e\mathbf{\Omega}_a-yy^T\mathbf{\Omega}_a\}=0\\
  \frac{\partial{Q}}{\partial{\sigma^2_e}} & =tr\{\frac{yy^Tyy^T-2(\sigma^2_a\mathbf{\Omega}_a+\sigma^2_eI_N)yy^T+(\sigma^2_g\mathbf{\Omega}_a+\sigma^2_eI_N)^2}{\partial{\sigma^2_e}}\}&=tr\{\sigma^2_a\mathbf{\Omega}_a+\sigma^2_e\mathbf{I}_N-yy^T\mathbf{I}_N\}=0
 \end{array}
\right.
\]

\[
\left\{
 \begin{array}{lll}
tr\{\sigma^2_a\mathbf{\Omega}_a\mathbf{\Omega}_a^T+\sigma^2_e\mathbf{\Omega}_a\}=tr\{yy^T\mathbf{\Omega}_a\}=tr\{y^T\mathbf{\Omega}_ay\}\\
tr\{\sigma^2_a\mathbf{\Omega}_a+\sigma^2_e\mathbf{I}_N\}=tr\{yy^T\mathbf{I}_N\}=tr\{y^T\mathbf{I}_Ny\}
 \end{array}
\right.
\] The last step uses cycling property of
\href{https://en.wikipedia.org/wiki/Trace_(linear_algebra)}{\(trace\)}.

The above result can be orgainzed in to normal equation below

\[
\begin{bmatrix} tr[\mathbf{\Omega}_a\mathbf{\Omega}_a^T] & tr[\mathbf{\Omega}_a] \\ tr[\mathbf{\Omega}_a] & N \end{bmatrix} \left[ \begin{array}{c} \hat{\sigma}^2_a \\ \hat{\sigma}^2_e \end{array} \right] = \left[ \begin{array}{c} tr[y^T\mathbf{\Omega}_ay] \\ tr[y^TI_Ny] \end{array} \right] 
\]

It is easy to find that

\[\color{green}{\boxed{\hat{\sigma}^2_a=\frac{y^T(\mathbf{\Omega}_a-I_N)y}{tr[\mathbf{\Omega}_a\mathbf{\Omega}_a^T]-N}}}\]

\textbf{Notes}:, the \(y^T(\mathbf{\Omega}-I_N)y\) is quadratic form
\href{https://en.wikipedia.org/wiki/Quadratic_form_(statistics)}{wiki
link}.

\(var[y^T(\mathbf{\Omega}_a-I_N)y]=2tr[H(\mathbf{\Omega}_a-I_N)H(\mathbf{\Omega}_a-I_N)]\)

in which \(H\) can be replaced by the expected correlation matrix for
\(y\), here is \(I_N\) for unrelated individuals. So,

\(var[y^T(\mathbf{\Omega}_a-I_N)y]=2tr[(\mathbf{\Omega}_a-I_N)(\mathbf{\Omega}_a-I_N)]=2tr[\mathbf{\Omega}^2_a-2\mathbf{\Omega}_aI_N-I_N^2]=2[tr(\mathbf{\Omega}_a^2)-N]=2N^2M_e\).
In addition,
\(E[y^T(\mathbf{\Omega}_a-I_N)y]=[tr(\mathbf{\Omega}_a^2)-N]\sigma^2_a\).

\begin{Shaded}
\begin{Highlighting}[]
\KeywordTok{sourceCpp}\NormalTok{(}\StringTok{"~/git/Notes/R/RLib/Shotgun.cpp"}\NormalTok{)}
\NormalTok{M=}\DecValTok{10000}
\NormalTok{N=}\DecValTok{200}
\NormalTok{frq=}\KeywordTok{runif}\NormalTok{(M, }\FloatTok{0.2}\NormalTok{, }\FloatTok{0.8}\NormalTok{)}
\NormalTok{Dp=}\KeywordTok{c}\NormalTok{(}\KeywordTok{runif}\NormalTok{(M}\OperatorTok{/}\DecValTok{2}\NormalTok{, }\FloatTok{-0.95}\NormalTok{, }\FloatTok{-0.5}\NormalTok{), }\KeywordTok{runif}\NormalTok{(M}\OperatorTok{/}\DecValTok{2}\NormalTok{, }\FloatTok{0.5}\NormalTok{, }\FloatTok{0.95}\NormalTok{))}
\NormalTok{Dp=Dp[}\DecValTok{1}\OperatorTok{:}\NormalTok{(M}\DecValTok{-1}\NormalTok{)]}

\NormalTok{Amat=}\KeywordTok{matrix}\NormalTok{(}\DecValTok{0}\NormalTok{, }\DecValTok{100}\NormalTok{, }\DecValTok{2}\NormalTok{)}
\NormalTok{gMat=}\KeywordTok{GenerateGenoDprimeRcpp}\NormalTok{(frq, }\DataTypeTok{N=}\NormalTok{N, }\DataTypeTok{Dp =}\NormalTok{ Dp)}

\NormalTok{sG=}\KeywordTok{scale}\NormalTok{(gMat)}
\NormalTok{K=sG}\OperatorTok\KeywordTok{t}\NormalTok{(sG)}\OperatorTok{/}\NormalTok{(M}\DecValTok{-1}\NormalTok{)}
\NormalTok{Me=}\DecValTok{1}\OperatorTok{/}\KeywordTok{var}\NormalTok{(K[}\KeywordTok{row}\NormalTok{(K)}\OperatorTok{<}\KeywordTok{col}\NormalTok{(K)])}
\NormalTok{I=}\KeywordTok{diag}\NormalTok{(}\DecValTok{1}\NormalTok{, N, N)}

\NormalTok{KI=K}\OperatorTok{-}\NormalTok{I}

\ControlFlowTok{for}\NormalTok{(i }\ControlFlowTok{in} \DecValTok{1}\OperatorTok{:}\DecValTok{100}\NormalTok{) \{}
\NormalTok{  y=}\KeywordTok{scale}\NormalTok{(}\KeywordTok{rnorm}\NormalTok{(N))}
\NormalTok{  yy=y}\OperatorTok\KeywordTok{t}\NormalTok{(y)}
\NormalTok{  yy_KI=yy}\OperatorTok\NormalTok{KI}

\NormalTok{  Amat[i,}\DecValTok{1}\NormalTok{]=}\KeywordTok{t}\NormalTok{(y)}\OperatorTok\NormalTok{KI}\OperatorTok\NormalTok{y}
\NormalTok{  ss=I}\OperatorTok\NormalTok{KI}\OperatorTok\NormalTok{I}\OperatorTok\NormalTok{KI}
\CommentTok{#  ss=KI%*%KI}
\NormalTok{  Amat[i,}\DecValTok{2}\NormalTok{]=}\DecValTok{2}\OperatorTok{*}\KeywordTok{sum}\NormalTok{(}\KeywordTok{diag}\NormalTok{(ss))}
\NormalTok{\}}
\KeywordTok{colMeans}\NormalTok{(Amat)}
\end{Highlighting}
\end{Shaded}

\begin{verbatim}
## [1] -0.6477451 15.4409693
\end{verbatim}

\begin{Shaded}
\begin{Highlighting}[]
\KeywordTok{print}\NormalTok{(}\KeywordTok{paste}\NormalTok{(}\StringTok{"Obs v:"}\NormalTok{, }\KeywordTok{var}\NormalTok{(Amat[,}\DecValTok{1}\NormalTok{]), }\StringTok{", Exp v:"}\NormalTok{, (}\DecValTok{2}\OperatorTok{*}\NormalTok{N}\OperatorTok{^}\DecValTok{2}\OperatorTok{/}\NormalTok{Me)))}
\end{Highlighting}
\end{Shaded}

\begin{verbatim}
## [1] "Obs v: 13.4777957244701 , Exp v: 13.4109808880952"
\end{verbatim}

\hypertarget{see-a-simulation-example-additive}{%
\paragraph{See a simulation example
(additive)}\label{see-a-simulation-example-additive}}

\begin{Shaded}
\begin{Highlighting}[]
\NormalTok{n=}\DecValTok{500} \CommentTok{#sample size}
\NormalTok{m=}\DecValTok{1000} \CommentTok{#marker}
\NormalTok{h2=}\FloatTok{0.3} \CommentTok{#heritability}
\NormalTok{b=}\KeywordTok{rnorm}\NormalTok{(m, }\DecValTok{0}\NormalTok{, }\KeywordTok{sqrt}\NormalTok{(h2}\OperatorTok{/}\NormalTok{m)) }\CommentTok{#effect}
\NormalTok{SIMU=}\DecValTok{5}

\CommentTok{#simu g}
\NormalTok{fq=}\KeywordTok{runif}\NormalTok{(m, }\FloatTok{0.1}\NormalTok{, }\FloatTok{0.5}\NormalTok{)}
\NormalTok{x=}\KeywordTok{matrix}\NormalTok{(}\DecValTok{0}\NormalTok{, n, m)}
\ControlFlowTok{for}\NormalTok{(i }\ControlFlowTok{in} \DecValTok{1}\OperatorTok{:}\NormalTok{m) \{}
\NormalTok{  x[,i]=}\KeywordTok{rbinom}\NormalTok{(n, }\DecValTok{2}\NormalTok{, fq[i])}
\NormalTok{\}}
\NormalTok{sx=}\KeywordTok{apply}\NormalTok{(x, }\DecValTok{2}\NormalTok{, scale)}

\NormalTok{K=sx}\OperatorTok\KeywordTok{t}\NormalTok{(sx)}\OperatorTok{/}\NormalTok{m}
\NormalTok{me=}\KeywordTok{var}\NormalTok{(K[}\KeywordTok{col}\NormalTok{(K)}\OperatorTok{<}\KeywordTok{row}\NormalTok{(K)])}

\CommentTok{#simu y}
\NormalTok{H2=}\KeywordTok{matrix}\NormalTok{(}\DecValTok{0}\NormalTok{, SIMU, }\DecValTok{2}\NormalTok{)}
\ControlFlowTok{for}\NormalTok{(i }\ControlFlowTok{in} \DecValTok{1}\OperatorTok{:}\NormalTok{SIMU) \{}
\NormalTok{  y=sx}\OperatorTok\NormalTok{b}\OperatorTok{+}\KeywordTok{rnorm}\NormalTok{(n, }\DecValTok{0}\NormalTok{, }\KeywordTok{sqrt}\NormalTok{(}\DecValTok{1}\OperatorTok{-}\NormalTok{h2))}

\NormalTok{  yy=y}\OperatorTok\KeywordTok{t}\NormalTok{(y)}
\NormalTok{  h2Mod=}\KeywordTok{lm}\NormalTok{(yy[}\KeywordTok{col}\NormalTok{(yy)}\OperatorTok{<}\KeywordTok{row}\NormalTok{(yy)]}\OperatorTok{~}\NormalTok{K[}\KeywordTok{col}\NormalTok{(yy)}\OperatorTok{<}\KeywordTok{row}\NormalTok{(yy)])}
\NormalTok{  H2[i,}\DecValTok{1}\NormalTok{]=}\KeywordTok{summary}\NormalTok{(h2Mod)}\OperatorTok{$}\NormalTok{coefficients[}\DecValTok{2}\NormalTok{,}\DecValTok{1}\NormalTok{]}
\NormalTok{  ss=}\KeywordTok{matrix}\NormalTok{(}\DecValTok{0}\NormalTok{, m, }\DecValTok{5}\NormalTok{)}
  \ControlFlowTok{for}\NormalTok{(j }\ControlFlowTok{in} \DecValTok{1}\OperatorTok{:}\NormalTok{m) \{}
\NormalTok{    mod=}\KeywordTok{lm}\NormalTok{(y}\OperatorTok{~}\NormalTok{x[,j])}
\NormalTok{    ss[j,}\DecValTok{1}\OperatorTok{:}\DecValTok{4}\NormalTok{]=}\KeywordTok{summary}\NormalTok{(mod)}\OperatorTok{$}\NormalTok{coefficient[}\DecValTok{2}\NormalTok{,]}
\NormalTok{  \}}
\NormalTok{  ss[,}\DecValTok{5}\NormalTok{]=ss[,}\DecValTok{3}\NormalTok{]}\OperatorTok{^}\DecValTok{2}
\NormalTok{  H2[i,}\DecValTok{2}\NormalTok{]=((}\KeywordTok{mean}\NormalTok{(ss[,}\DecValTok{5}\NormalTok{])}\OperatorTok{-}\DecValTok{1}\NormalTok{)}\OperatorTok{*}\NormalTok{n)}\OperatorTok{/}\NormalTok{(n}\OperatorTok{*}\NormalTok{n}\OperatorTok{*}\NormalTok{me)}
\NormalTok{\}}
\KeywordTok{barplot}\NormalTok{(}\KeywordTok{t}\NormalTok{(H2), }\DataTypeTok{beside =}\NormalTok{ T)}
\KeywordTok{abline}\NormalTok{(}\DataTypeTok{h=}\NormalTok{h2)}
\end{Highlighting}
\end{Shaded}

\includegraphics{H2ss_files/figure-latex/road-test-1.pdf}

\hypertarget{dominance-vc}{%
\subsubsection{Dominance VC}\label{dominance-vc}}

\(y|e,\mathbf{\beta}=\mathbf{X}_a\beta_a+\mathbf{X}_d\beta_d+e\)

\(e|\sigma^2_e~N(0,\sigma^2I_N)\)

\(\beta_a|\sigma^2_a~N(\frac{\sigma^2_a}{M},I_M)\)

\(\beta_d|\sigma^2_d~N(\frac{\sigma^2_d}{M},I_M)\)

\(y\) is centered.

\[
\begin{align}
cov(y)&=\mathbb{E}(yy^T)-\mathbb{y}\mathbb{y}^T\\
&=\sigma^2_a\mathbf{\Omega}_a+\sigma^2_d\mathbf{\Omega}_d+\sigma^2_eI_N
\end{align}
\]

When include dominace variance component, the target to be minimized is

\(Q=argmin_{(\sigma^2_g,\sigma^2_d,\sigma^2_e)}|| yy^T-(\sigma^2_a\mathbf{\Omega}_a+\sigma^2_d\mathbf{\Omega}_d+\sigma^2_eI_N)||_F^2\)

To minimize the target function below

\[
\begin{align}
Q&=argmin_{(\sigma^2_a,\sigma^2_d,\sigma^2_e)}|| yy^T-(\sigma^2_a\mathbf{\Omega}_a+\sigma^2_d\mathbf{\Omega}_d+\sigma^2_eI_N)||_F^2\\
&=tr\{[yy^T-(\sigma^2_a\mathbf{\Omega}_a+\sigma^2_d\mathbf{\Omega}_d+\sigma^2_eI_N)][yy^T-(\sigma^2_a\mathbf{\Omega}_a+\sigma^2_d\mathbf{\Omega}_d+\sigma^2_eI_N)]^T\}
\end{align}
\] After taking deriation in repect to each component, we have \[
\left\{
 \begin{array}{lll}
  \frac{\partial{Q}}{\partial{\sigma^2_a}} & =tr\{\frac{yy^Tyy^T-2(\sigma^2_a\mathbf{\Omega}_a+\sigma^2_d\mathbf{\Omega}_d+\sigma^2_eI_N)yy^T+(\sigma^2_a\mathbf{\Omega}_a+\sigma^2_d\mathbf{\Omega}_d+\sigma^2_eI_N)^2}{\partial{\sigma^2_a}}\}&=tr\{\sigma^2_a\mathbf{\Omega}_a\mathbf{\Omega}_a^T+\sigma^2_d\mathbf{\Omega}_a\mathbf{\Omega}_d^T+\sigma^2_e\mathbf{\Omega}_a-yy^T\mathbf{\Omega}_a\}=0\\
  \frac{\partial{Q}}{\partial{\sigma^2_d}} & =tr\{\frac{yy^Tyy^T-2(\sigma^2_a\mathbf{\Omega}_a+\sigma^2_d\mathbf{\Omega}_d+\sigma^2_eI_N)yy^T+(\sigma^2_a\mathbf{\Omega}_a+\sigma^2_d\mathbf{\Omega}_d+\sigma^2_eI_N)^2}{\partial{\sigma^2_d}}\}&=tr\{\sigma^2_d\mathbf{\Omega}_d\mathbf{\Omega}_a^T+\sigma^2_d\mathbf{\Omega}_d\mathbf{\Omega}_d^T+\sigma^2_e\mathbf{\Omega}_a-yy^T\mathbf{\Omega}_d\}=0\\
  \frac{\partial{Q}}{\partial{\sigma^2_e}} & =tr\{\frac{yy^Tyy^T-2(\sigma^2_a\mathbf{\Omega}_a+\sigma^2_d\mathbf{\Omega}_d+\sigma^2_eI_N)yy^T+(\sigma^2_a\mathbf{\Omega}_a+\sigma^2_d\mathbf{\Omega}_d+\sigma^2_eI_N)^2}{\partial{\sigma^2_e}}\}&=tr\{\sigma^2_a\mathbf{\Omega}_a+\sigma^2_d\mathbf{\Omega}_d+\sigma^2_e\mathbf{I}_N-yy^T\mathbf{I}_N\}=0
 \end{array}
\right.
\] then \[
\left\{
 \begin{array}{lll}
tr\{\sigma^2_a\mathbf{\Omega}_a\mathbf{\Omega}_a^T+\sigma^2_a\mathbf{\Omega}_a\mathbf{\Omega}_d^T+\sigma^2_e\mathbf{\Omega}_a\}=tr\{yy^T\mathbf{\Omega}_a\}=tr\{y^T\mathbf{\Omega}_ay\}\\
tr\{\sigma^2_d\mathbf{\Omega}_d\mathbf{\Omega}_a^T+\sigma^2_d\mathbf{\Omega}_d\mathbf{\Omega}_d^T+\sigma^2_e\mathbf{\Omega}_d\}=tr\{yy^T\mathbf{\Omega}_d\}=tr\{y^T\mathbf{\Omega}_dy\}\\
tr\{\sigma^2_a\mathbf{\Omega}_a+\sigma^2_d\mathbf{\Omega}_d+\sigma^2_e\mathbf{I}_N\}=tr\{yy^T\mathbf{I}_N\}=tr\{y^T\mathbf{I}_Ny\}
 \end{array}
\right.
\]

We need to solve the equation below \[
\begin{bmatrix} 
tr[\mathbf{\Omega}_a\mathbf{\Omega}_a^T] & tr[\mathbf{\Omega}_a\mathbf{\Omega}_d^T]& tr[\mathbf{\Omega}_a] \\
tr[\mathbf{\Omega}_d\mathbf{\Omega}_a^T] & tr[\mathbf{\Omega}_d\mathbf{\Omega}_d^T] & tr[\mathbf{\Omega}_d] \\
tr[\mathbf{\Omega}_a] & tr[\mathbf{\Omega}_d] & N \end{bmatrix} \left[ \begin{array}{c} \hat{\sigma}^2_a \\ 
\hat{\sigma}^2_d \\
\hat{\sigma}^2_e \end{array} \right] = 
\left[ \begin{array}{c} tr[y^T\mathbf{\Omega}_ay] \\ tr[y^T\mathbf{\Omega}_dy]\\tr[y^TI_Ny] \end{array} \right] 
\] Of note, \(E[tr(\mathbf{\Omega})_a]=E[tr(\mathbf{\Omega})_d]=N\).

\[
\left[ \begin{array}{c}
\hat{\sigma}^2_a \\
\hat{\sigma}^2_d \\
\hat{\sigma}^2_e \\
\end{array}
\right ]=\Lambda^{-1}
\left[ \begin{array}{c} y^T\mathbf{\Omega}_ay \\ y^T\mathbf{\Omega}_dy\\ y^TI_Ny \end{array} \right] 
\]

In particular, \[
\begin{align}
\hat{\sigma}^2_a&=&\frac{(ei-fh)y^T\mathbf{\Omega}_ay-(bi-ch)y^T\mathbf{\Omega}_dy+(bf-ce)y^TI_Ny}{det(\mathbf{\Lambda})}\\
&=&\frac{1}{det(\mathbf{\Lambda})}\{[Ntr(\mathbf \Omega^2_d)-N^2]y^T\mathbf{\Omega}_ay-[Ntr(\mathbf \Omega^2_d)-N^2]y^TI_Ny\}\\
&=&\frac{1}{det(\mathbf{\Lambda})}\{[Ntr(\mathbf \Omega^2_d)-N^2]y^T(\mathbf{\Omega}_a-I_N)y\}
\end{align}
\],

in which \[
\begin{align}
det(\mathbf{\Lambda})&=&abd+bfg+cdh-ceg-fha-ibd\\
&=&\color{red}{tr(\mathbf{\Omega}^2_a}\color{blue}{\mathbf{\Omega}^2_d})N+N^3+N^3-N^2tr(\mathbf{\Omega}^2_a)-N^2tr(\mathbf{\Omega}^2_d)-N^3\\
&=&N[\color{red}{tr(\mathbf{\Omega^2}_a} \color{blue}{\mathbf{\Omega^2}_d)}-tr(\mathbf{\Omega^2}_a)N-tr(\mathbf{\Omega}^2_d)N+N^2]
\end{align}
\].

Under orthogonal coding for additive and dominance scheme,
\(\color{red}{tr(\mathbf{\Omega}^2_a}\color{blue}{\mathbf{\Omega}^2_d)}=\color{red}{tr(\mathbf{\Omega}^2_a)}\color{blue}{tr{(\mathbf{\Omega}^2_d)}}\),
and
\(det(\mathbf {\Lambda})=N[\color{red}{tr(\mathbf{\Omega}^2_a)}-N][\color{blue}{tr(\mathbf{\Omega}^2_d)}-N]\)

Then,
\(\hat{\sigma}^2_a=\frac{y^T[\mathbf{\Omega}_a-I_N]y}{tr(\mathbf{\Omega}^2_a)-N)}\).

Similarly for,
\(\hat{\sigma}^2_d=\frac{y^T[\mathbf{\Omega}_d-I_N]y}{tr(\mathbf{\Omega}^2_d)-N)}\).

\hypertarget{see-a-dominace-example}{%
\subsubsection{See a dominace example}\label{see-a-dominace-example}}

\begin{Shaded}
\begin{Highlighting}[]
\NormalTok{n=}\DecValTok{500} \CommentTok{#sample size}
\NormalTok{m=}\DecValTok{1000} \CommentTok{#marker}
\NormalTok{h2=}\FloatTok{0.3} \CommentTok{#heritability}
\NormalTok{h2d=}\FloatTok{0.3}
\NormalTok{b=}\KeywordTok{rnorm}\NormalTok{(m, }\DecValTok{0}\NormalTok{, }\KeywordTok{sqrt}\NormalTok{(h2}\OperatorTok{/}\NormalTok{m)) }\CommentTok{#effect}
\NormalTok{d=}\KeywordTok{rnorm}\NormalTok{(m, }\DecValTok{0}\NormalTok{, }\KeywordTok{sqrt}\NormalTok{(h2d}\OperatorTok{/}\NormalTok{m))}
\NormalTok{SIMU=}\DecValTok{5}

\CommentTok{#simu g}
\NormalTok{fq=}\KeywordTok{runif}\NormalTok{(m, }\FloatTok{0.3}\NormalTok{, }\FloatTok{0.5}\NormalTok{)}
\NormalTok{x=}\KeywordTok{matrix}\NormalTok{(}\DecValTok{0}\NormalTok{, n, m)}
\ControlFlowTok{for}\NormalTok{(i }\ControlFlowTok{in} \DecValTok{1}\OperatorTok{:}\NormalTok{m) \{}
\NormalTok{  x[,i]=}\KeywordTok{rbinom}\NormalTok{(n, }\DecValTok{2}\NormalTok{, fq[i])}
\NormalTok{\}}
\NormalTok{FQ=}\KeywordTok{colMeans}\NormalTok{(x)}\OperatorTok{/}\DecValTok{2}
\NormalTok{sA=}\KeywordTok{apply}\NormalTok{(x, }\DecValTok{2}\NormalTok{, scale)}

\NormalTok{K=sA}\OperatorTok\KeywordTok{t}\NormalTok{(sA)}\OperatorTok{/}\NormalTok{m}
\NormalTok{me=}\KeywordTok{var}\NormalTok{(K[}\KeywordTok{col}\NormalTok{(K)}\OperatorTok{<}\KeywordTok{row}\NormalTok{(K)])}


\CommentTok{##dominance}
\NormalTok{xd=}\KeywordTok{matrix}\NormalTok{(}\DecValTok{0}\NormalTok{, n, m)}

\ControlFlowTok{for}\NormalTok{(i }\ControlFlowTok{in} \DecValTok{1}\OperatorTok{:}\NormalTok{m) \{}
\NormalTok{  cd=}\KeywordTok{c}\NormalTok{(}\DecValTok{0}\NormalTok{, }\DecValTok{2}\OperatorTok{*}\NormalTok{FQ[i], }\DecValTok{4}\OperatorTok{*}\NormalTok{FQ[i]}\OperatorTok{-}\DecValTok{2}\NormalTok{)}
\NormalTok{  xd[,i]=cd[x[,i]}\OperatorTok{+}\DecValTok{1}\NormalTok{]}
\NormalTok{\}}
\NormalTok{dCt=}\KeywordTok{matrix}\NormalTok{(}\KeywordTok{rep}\NormalTok{(}\DecValTok{2}\OperatorTok{*}\NormalTok{FQ}\OperatorTok{^}\DecValTok{2}\NormalTok{, n), n, m, }\DataTypeTok{byrow =}\NormalTok{ T)}
\NormalTok{dV=}\KeywordTok{matrix}\NormalTok{(}\KeywordTok{rep}\NormalTok{(}\KeywordTok{sqrt}\NormalTok{(}\DecValTok{4}\OperatorTok{*}\NormalTok{FQ}\OperatorTok{^}\DecValTok{2}\OperatorTok{*}\NormalTok{(}\DecValTok{1}\OperatorTok{-}\NormalTok{FQ)}\OperatorTok{^}\DecValTok{2}\NormalTok{), n), n, m, }\DataTypeTok{byrow =}\NormalTok{ T)}
\NormalTok{sD=(xd}\OperatorTok{-}\NormalTok{dCt)}\OperatorTok{/}\NormalTok{dV}
\NormalTok{Kd=sD}\OperatorTok\KeywordTok{t}\NormalTok{(sD)}\OperatorTok{/}\NormalTok{m}
\NormalTok{med=}\KeywordTok{var}\NormalTok{(Kd[}\KeywordTok{col}\NormalTok{(K)}\OperatorTok{<}\KeywordTok{row}\NormalTok{(K)])}

\NormalTok{Dm=}\KeywordTok{ifelse}\NormalTok{(x}\OperatorTok{==}\DecValTok{1}\NormalTok{, }\DecValTok{1}\NormalTok{, }\DecValTok{0}\NormalTok{)}
\CommentTok{#simu y}
\NormalTok{H2=}\KeywordTok{matrix}\NormalTok{(}\DecValTok{0}\NormalTok{, SIMU, }\DecValTok{4}\NormalTok{)}
\ControlFlowTok{for}\NormalTok{(i }\ControlFlowTok{in} \DecValTok{1}\OperatorTok{:}\NormalTok{SIMU) \{}
\NormalTok{  y=x}\OperatorTok\NormalTok{b}\OperatorTok{+}\NormalTok{Dm}\OperatorTok\NormalTok{d}
\NormalTok{  vy=}\KeywordTok{var}\NormalTok{(y)}
\NormalTok{  y=y}\OperatorTok{+}\KeywordTok{rnorm}\NormalTok{(n, }\DecValTok{0}\NormalTok{, }\KeywordTok{sqrt}\NormalTok{(vy}\OperatorTok{/}\NormalTok{(h2}\OperatorTok{+}\NormalTok{h2d)}\OperatorTok{*}\NormalTok{(}\DecValTok{1}\OperatorTok{-}\NormalTok{h2}\OperatorTok{-}\NormalTok{h2d)))}
\NormalTok{  y=}\KeywordTok{scale}\NormalTok{(y)}
\NormalTok{  yy=y}\OperatorTok\KeywordTok{t}\NormalTok{(y)}
\NormalTok{  h2Mod=}\KeywordTok{lm}\NormalTok{(yy[}\KeywordTok{col}\NormalTok{(yy)}\OperatorTok{<}\KeywordTok{row}\NormalTok{(yy)]}\OperatorTok{~}\NormalTok{K[}\KeywordTok{col}\NormalTok{(yy)}\OperatorTok{<}\KeywordTok{row}\NormalTok{(yy)]}\OperatorTok{+}\NormalTok{Kd[}\KeywordTok{col}\NormalTok{(yy)}\OperatorTok{<}\KeywordTok{row}\NormalTok{(yy)])}
\NormalTok{  H2[i,}\DecValTok{1}\NormalTok{]=}\KeywordTok{summary}\NormalTok{(h2Mod)}\OperatorTok{$}\NormalTok{coefficients[}\DecValTok{2}\NormalTok{,}\DecValTok{1}\NormalTok{]}
\NormalTok{  H2[i,}\DecValTok{2}\NormalTok{]=}\KeywordTok{summary}\NormalTok{(h2Mod)}\OperatorTok{$}\NormalTok{coefficients[}\DecValTok{3}\NormalTok{,}\DecValTok{1}\NormalTok{]}
\NormalTok{  ss=}\KeywordTok{matrix}\NormalTok{(}\DecValTok{0}\NormalTok{, m, }\DecValTok{5}\NormalTok{)}
\NormalTok{  ssd=}\KeywordTok{matrix}\NormalTok{(}\DecValTok{0}\NormalTok{, m, }\DecValTok{5}\NormalTok{)}
  \ControlFlowTok{for}\NormalTok{(j }\ControlFlowTok{in} \DecValTok{1}\OperatorTok{:}\NormalTok{m) \{}
\NormalTok{    mod=}\KeywordTok{lm}\NormalTok{(y}\OperatorTok{~}\NormalTok{sA[,j]}\OperatorTok{+}\NormalTok{sD[,j])}
\NormalTok{    ss[j,}\DecValTok{1}\OperatorTok{:}\DecValTok{4}\NormalTok{]=}\KeywordTok{summary}\NormalTok{(mod)}\OperatorTok{$}\NormalTok{coefficient[}\DecValTok{2}\NormalTok{,]}
\NormalTok{    ssd[j,}\DecValTok{1}\OperatorTok{:}\DecValTok{4}\NormalTok{]=}\KeywordTok{summary}\NormalTok{(mod)}\OperatorTok{$}\NormalTok{coefficient[}\DecValTok{3}\NormalTok{,]}
\NormalTok{  \}}
\NormalTok{  ss[,}\DecValTok{5}\NormalTok{]=ss[,}\DecValTok{3}\NormalTok{]}\OperatorTok{^}\DecValTok{2}
\NormalTok{  H2[i,}\DecValTok{3}\NormalTok{]=((}\KeywordTok{mean}\NormalTok{(ss[,}\DecValTok{5}\NormalTok{])}\OperatorTok{-}\DecValTok{1}\NormalTok{)}\OperatorTok{*}\NormalTok{n)}\OperatorTok{/}\NormalTok{(n}\OperatorTok{*}\NormalTok{n}\OperatorTok{*}\NormalTok{me)}
\NormalTok{  ssd[,}\DecValTok{5}\NormalTok{]=ssd[,}\DecValTok{3}\NormalTok{]}\OperatorTok{^}\DecValTok{2}
\NormalTok{  H2[i,}\DecValTok{4}\NormalTok{]=((}\KeywordTok{mean}\NormalTok{(ssd[,}\DecValTok{5}\NormalTok{])}\OperatorTok{-}\DecValTok{1}\NormalTok{)}\OperatorTok{*}\NormalTok{n)}\OperatorTok{/}\NormalTok{(n}\OperatorTok{*}\NormalTok{n}\OperatorTok{*}\NormalTok{med)}
\NormalTok{\}}
\KeywordTok{barplot}\NormalTok{(}\KeywordTok{t}\NormalTok{(H2), }\DataTypeTok{beside =}\NormalTok{ T)}
\KeywordTok{abline}\NormalTok{(}\DataTypeTok{h=}\KeywordTok{c}\NormalTok{(h2, h2d))}
\end{Highlighting}
\end{Shaded}

\includegraphics{H2ss_files/figure-latex/dom-1.pdf}

\hypertarget{link-to-summary-statistics}{%
\subsubsection{Link to summary
statistics}\label{link-to-summary-statistics}}

After rearrangement, it is easy to see that
\(y^T\mathbf{\Omega}y=\frac{1}{M}y^T\mathbf{SS}^Ty\), and
\(E(\frac{1}{M}y^T\mathbf{SS}^Ty)=NE(\chi^2_{1,h^2})=N(1+N\frac{h^2}{M_Q}\sum_{m=1}^Mr^2_{k,m})\).
\texttt{**It\ is\ available\ as\ summary\ statistics!**}. If further
adjustment for summary statistics is requested, a modified LSE can be
used to ajust summary statistics (see LSE phenome).

It is easy to see \(E(y^TI_Ny)=N\), so the expectation of the numerator
is \(N^2\frac{h^2}{M_q}\sum^{M_{q}}_{m=1}r^2_{k,m}=N^2h^2E(\bar{r}^2)\),
in which \(M_q\) is the number of causal variants.

\hypertarget{link-to-the-reference-population}{%
\subsubsection{Link to the reference
population}\label{link-to-the-reference-population}}

The denominator can be derived that
\(tr[\mathbf{\Omega}^2]=N^2E^2(\rho)+N\), in which
\[E^2(\rho)=\frac{\sum_{k_1=1}^M\sum_{k_2=1}^M \rho^2_{k_1k_2}}{M^2} = \frac{\sum_{k=1}^M1+\sum_{k_1=1}^M\sum_{k_2 \ne k_1}^M \rho^2_{k_1k_2}}{M^2} \geq \frac{1}{M}\]

Or, \[\frac{[E(\sum_i^M\chi^2_1)-1]N}{N^2/M_e}\]

\hypertarget{ukb-case}{%
\subsubsection{UKB case}\label{ukb-case}}

A realized example please refer to
\(\color{red} {/public/home/xuhm/gc5k/UKB\_cohort/ss}\),
\(\color{green} {/public/home/xuhm/gc5k/UKB\_cohort/ss\_adj}\)

\begin{Shaded}
\begin{Highlighting}[]
\KeywordTok{library}\NormalTok{(}\StringTok{"knitr"}\NormalTok{)}
\KeywordTok{library}\NormalTok{(}\StringTok{"kableExtra"}\NormalTok{)}

\NormalTok{ch2=}\KeywordTok{read.table}\NormalTok{(}\StringTok{"cohortH2.txt"}\NormalTok{, }\DataTypeTok{as.is =}\NormalTok{ T, }\DataTypeTok{header =}\NormalTok{ T)}
\NormalTok{knitr}\OperatorTok{::}\KeywordTok{kable}\NormalTok{(ch2, }\DataTypeTok{caption =} \StringTok{"cohort h2"}\NormalTok{) }\OperatorTok
\KeywordTok{kable_styling}\NormalTok{(}\StringTok{"striped"}\NormalTok{, }\DataTypeTok{full_width =}\NormalTok{ T) }\OperatorTok
\KeywordTok{row_spec}\NormalTok{(}\DataTypeTok{row=}\DecValTok{16}\OperatorTok{:}\DecValTok{16}\NormalTok{, }\DataTypeTok{color=}\StringTok{"white"}\NormalTok{, }\DataTypeTok{background=}\StringTok{"red"}\NormalTok{)}
\end{Highlighting}
\end{Shaded}

\begin{table}

\caption{\label{tab:cohorth2}cohort h2}
\centering
\begin{tabu} to \linewidth {>{\raggedright}X>{\raggedleft}X>{\raggedleft}X>{\raggedleft}X>{\raggedleft}X>{\raggedleft}X>{\raggedleft}X}
\hline
Cohort & N & Me & chisq & h2 & chisq.1 & h2.adjust\_sex.\\
\hline
Barts & 7699 & 51936.00 & 1.116 & 0.794 & 1.2230 & 1.5222\\
\hline
Birmingham & 19529 & 55594.92 & 1.100 & 0.287 & 1.2102 & 0.6050\\
\hline
Bristol & 39172 & 56774.84 & 1.218 & 0.320 & 1.4353 & 0.6380\\
\hline
Bury & 25404 & 55092.41 & 1.154 & 0.338 & 1.3046 & 0.6686\\
\hline
Cardiff & 16226 & 52069.39 & 1.111 & 0.361 & 1.2325 & 0.7549\\
\hline
Croydon & 18475 & 57405.47 & 1.107 & 0.337 & 1.2260 & 0.7099\\
\hline
Edinburgh & 15441 & 53041.79 & 1.111 & 0.386 & 1.2156 & 0.7490\\
\hline
Glasgow & 16299 & 51940.17 & 1.156 & 0.502 & 1.2900 & 0.9351\\
\hline
Hounslow & 18420 & 56773.33 & 1.133 & 0.414 & 1.2382 & 0.7427\\
\hline
Leeds & 39651 & 56474.83 & 1.242 & 0.349 & 1.4597 & 0.6623\\
\hline
Liverpool & 29604 & 54490.45 & 1.183 & 0.341 & 1.3665 & 0.6826\\
\hline
Manchester & 11502 & 53621.53 & 1.099 & 0.468 & 1.1650 & 0.7782\\
\hline
Middlesborough & 19819 & 53409.79 & 1.133 & 0.362 & 1.2228 & 0.6074\\
\hline
Newcastle & 34663 & 54675.71 & 1.179 & 0.286 & 1.3564 & 0.5686\\
\hline
Nottingham & 30760 & 56132.08 & 1.173 & 0.319 & 1.3407 & 0.6287\\
\hline
\rowcolor{red}  \textcolor{white}{Oxford} & \textcolor{white}{12260} & \textcolor{white}{55143.14} & \textcolor{white}{1.072} & \textcolor{white}{0.328} & \textcolor{white}{1.1345} & \textcolor{white}{0.6112}\\
\hline
Reading & 25886 & 58027.47 & 1.109 & 0.247 & 1.2400 & 0.5440\\
\hline
Sheffield & 27745 & 55244.01 & 1.145 & 0.293 & 1.3013 & 0.6069\\
\hline
Stoke & 18211 & 51302.58 & 1.114 & 0.326 & 1.2534 & 0.7220\\
\hline
\end{tabu}
\end{table}

\hypertarget{oxford}{%
\subsubsection{Oxford}\label{oxford}}

\texttt{/public/home/xuhm/gc5k/UKB\_cohort/ss\_chr}

\begin{Shaded}
\begin{Highlighting}[]
\KeywordTok{library}\NormalTok{(}\StringTok{"knitr"}\NormalTok{)}
\KeywordTok{library}\NormalTok{(}\StringTok{"kableExtra"}\NormalTok{)}
\NormalTok{ch2=}\KeywordTok{read.table}\NormalTok{(}\StringTok{"oxChr.txt"}\NormalTok{, }\DataTypeTok{as.is =}\NormalTok{ T, }\DataTypeTok{header =}\NormalTok{ T)}
\NormalTok{knitr}\OperatorTok{::}\KeywordTok{kable}\NormalTok{(ch2, }\DataTypeTok{caption =} \StringTok{"Oxford h2"}\NormalTok{) }\OperatorTok
\KeywordTok{kable_styling}\NormalTok{(}\StringTok{"striped"}\NormalTok{, }\DataTypeTok{full_width =}\NormalTok{ T) }\OperatorTok
\KeywordTok{row_spec}\NormalTok{(}\DataTypeTok{row=}\DecValTok{6}\OperatorTok{:}\DecValTok{6}\NormalTok{, }\DataTypeTok{bold=}\NormalTok{T, }\DataTypeTok{color =} \StringTok{"white"}\NormalTok{, }\DataTypeTok{background =} \StringTok{"red"}\NormalTok{)}
\end{Highlighting}
\end{Shaded}

\begin{table}

\caption{\label{tab:oxChr}Oxford h2}
\centering
\begin{tabu} to \linewidth {>{\raggedleft}X>{\raggedleft}X>{\raggedleft}X>{\raggedleft}X>{\raggedleft}X}
\hline
Chr & M & Me & chisq & h2\\
\hline
1 & 32913 & 7204.5646 & 1.0755 & 0.04480\\
\hline
2 & 33622 & 6403.3527 & 1.0321 & 0.01700\\
\hline
3 & 28040 & 5839.5143 & 1.0861 & 0.04140\\
\hline
4 & 26836 & 5632.8855 & 1.0545 & 0.02530\\
\hline
5 & 25243 & 5368.6945 & 1.0443 & 0.01960\\
\hline
\rowcolor{red}  \textcolor{white}{\textbf{6}} & \textcolor{white}{\textbf{30405}} & \textcolor{white}{\textbf{857.1236}} & \textcolor{white}{\textbf{1.3061}} & \textcolor{white}{\textbf{0.02170}}\\
\hline
7 & 23260 & 5020.4250 & 1.0346 & 0.01433\\
\hline
8 & 21530 & 3909.6782 & 1.0736 & 0.02370\\
\hline
9 & 18799 & 4277.3328 & 1.0470 & 0.01660\\
\hline
10 & 21313 & 4328.9556 & 1.0324 & 0.01160\\
\hline
11 & 20746 & 3269.6384 & 1.0349 & 0.00940\\
\hline
12 & 20194 & 4213.4118 & 1.0714 & 0.02480\\
\hline
13 & 14769 & 3508.2313 & 1.0363 & 0.01050\\
\hline
14 & 13741 & 3106.6733 & 1.0716 & 0.01830\\
\hline
15 & 13502 & 2872.9585 & 1.0998 & 0.02360\\
\hline
16 & 14848 & 3146.7089 & 1.0180 & 0.00470\\
\hline
17 & 14267 & 1796.0199 & 1.0360 & 0.00530\\
\hline
18 & 12880 & 3243.4674 & 1.0560 & 0.01500\\
\hline
19 & 12394 & 2346.5845 & 1.0938 & 0.01820\\
\hline
20 & 11181 & 2683.5395 & 1.0824 & 0.01820\\
\hline
21 & 6532 & 2683.5395 & 1.0076 & 0.00170\\
\hline
22 & 7087 & 1597.1183 & 1.0161 & 0.00210\\
\hline
\end{tabu}
\end{table}

\hypertarget{grm-statistics-mathbfomega_a}{%
\subsection{\texorpdfstring{2 GRM statistics
\(\mathbf{\Omega}_a\)}{2 GRM statistics \textbackslash{}mathbf\{\textbackslash{}Omega\}\_a}}\label{grm-statistics-mathbfomega_a}}

\(\mathbf{\Omega}=\mathbf{s^Ts}\), in which \(\mathbf{s}\) as defined in
the last page. In particular, for a pair of individuals \(i\) and \(j\),
their relatedness is
\(\Omega_{ij}=\frac{1}{M}\sum_{k=1}^Ms_{i,k}s_{j,k}=\frac{1}{M}\sum_{k=1}^Ms_{i,k}s_{j,k}\)

\(\mathbf{\Omega}\) is a Hermitian matrix (symetric), and the diagonal
element is \(\Omega_{ii}=\frac{1}{M}\sum_{k=1}^Ms_{i,k}^2\).

It exist simple relationship that
\(M_e=var(\Omega_{off-diag})=\frac{M+\sum_{l_1 \ne l_2}r_{l_1l_2}^2}{M^2}=\),
and \(E(V(M_e))=\frac{2M_e}{n^2}\)

\begin{Shaded}
\begin{Highlighting}[]
\KeywordTok{sourceCpp}\NormalTok{(}\StringTok{"~/git/Notes/R/RLib/Shotgun.cpp"}\NormalTok{)}
\NormalTok{M=}\DecValTok{1000}
\NormalTok{N=}\DecValTok{100}
\NormalTok{frq=}\KeywordTok{runif}\NormalTok{(M, }\FloatTok{0.2}\NormalTok{, }\FloatTok{0.8}\NormalTok{)}
\NormalTok{Dp=}\KeywordTok{c}\NormalTok{(}\KeywordTok{runif}\NormalTok{(M}\OperatorTok{/}\DecValTok{2}\NormalTok{, }\FloatTok{-0.95}\NormalTok{, }\FloatTok{-0.5}\NormalTok{), }\KeywordTok{runif}\NormalTok{(M}\OperatorTok{/}\DecValTok{2}\NormalTok{, }\FloatTok{0.5}\NormalTok{, }\FloatTok{0.95}\NormalTok{))}
\NormalTok{Dp=Dp[}\DecValTok{1}\OperatorTok{:}\NormalTok{(M}\DecValTok{-1}\NormalTok{)]}

\NormalTok{Amat=}\KeywordTok{matrix}\NormalTok{(}\DecValTok{0}\NormalTok{, }\DecValTok{100}\NormalTok{, }\DecValTok{2}\NormalTok{)}

\NormalTok{MeMat=}\KeywordTok{matrix}\NormalTok{(}\DecValTok{0}\NormalTok{, }\DecValTok{100}\NormalTok{, }\DecValTok{1}\NormalTok{)}
\ControlFlowTok{for}\NormalTok{(i }\ControlFlowTok{in} \DecValTok{1}\OperatorTok{:}\DecValTok{100}\NormalTok{) \{}
\NormalTok{gMat=}\KeywordTok{GenerateGenoDprimeRcpp}\NormalTok{(frq, }\DataTypeTok{N=}\NormalTok{N, }\DataTypeTok{Dp =}\NormalTok{ Dp)}

\NormalTok{sG=}\KeywordTok{scale}\NormalTok{(gMat)}
\NormalTok{K=sG}\OperatorTok\KeywordTok{t}\NormalTok{(sG)}\OperatorTok{/}\NormalTok{(M}\DecValTok{-1}\NormalTok{)}
\NormalTok{MeMat[i,}\DecValTok{1}\NormalTok{]=}\DecValTok{1}\OperatorTok{/}\KeywordTok{var}\NormalTok{(K[}\KeywordTok{row}\NormalTok{(K)}\OperatorTok{>}\KeywordTok{col}\NormalTok{(K)])}
\NormalTok{\}}
\KeywordTok{print}\NormalTok{(}\KeywordTok{paste}\NormalTok{(}\StringTok{"Me:"}\NormalTok{, }\KeywordTok{mean}\NormalTok{(MeMat)))}
\end{Highlighting}
\end{Shaded}

\begin{verbatim}
## [1] "Me: 605.431044673586"
\end{verbatim}

\begin{Shaded}
\begin{Highlighting}[]
\KeywordTok{print}\NormalTok{(}\KeywordTok{paste}\NormalTok{(}\StringTok{"V(Me):"}\NormalTok{, }\KeywordTok{sd}\NormalTok{(MeMat), }\StringTok{"E[V(Me)]:"}\NormalTok{, (}\DecValTok{2}\OperatorTok{/}\NormalTok{N}\OperatorTok{*}\KeywordTok{mean}\NormalTok{(MeMat))))}
\end{Highlighting}
\end{Shaded}

\begin{verbatim}
## [1] "V(Me): 12.9599221762403 E[V(Me)]: 12.1086208934717"
\end{verbatim}

\hypertarget{diagonal-elements}{%
\subsubsection{Diagonal elements}\label{diagonal-elements}}

\(E(s_{ii}^2)=diag(\Omega_{ii})=1\), interpreting that an individual is
identical to himself under random mating (no inbred, otherwise \(1+F\)).

\begin{align}
E(\Omega_{ii}^2) &= \frac{1}{M^2}[\color{red} {\sum_{k=1}^M s_{ik}^4}+\color{green}{\sum_{k_1}\sum_{k_1 \ne k_2} s_{i,k_1}^2s_{i,k_2}^2}] \\
 &=\frac{1}{M^2}\{\color{red} {\sum_{k=1}^M \frac{1}{2p_kq_k}}+\color{green}{\sum_{k_1}\sum_{k_1 \ne k_2} [(1+\rho^2_{k_1,k_2})+\frac{\rho_{k_1,k_2}(p_{k_1}-q_{k_1})(p_{k_2}-q_{k_2})}{2\sqrt{p_{k_1}q_{k_1}p_{k_2}q_{k_2}}}]}\} 
\end{align}

Under large sample and many loci,
\(E(\frac{\rho_{k_1,k_2}(p_{k_1}-q_{k_1})(p_{k_2}-q_{k_2})}{2\sqrt{p_{k_1}q_{k_1}p_{k_2}q_{k_2}}})=0\)

So, we have

\begin{align}
E(\Omega_{ii}^2)&=\frac{1}{M^2}[M\frac{1}{E(\sigma_k^2)}+M(M-1)(1+E(\rho^2_{k_1k_2}))]\\
&=\frac{1}{E(\sigma_k^2)}\frac{1}{M}+\frac{M-1}{M}[1+\color{red}{E(\rho^2_{k_1k_2})}]
\end{align}

\hypertarget{off-diagonal-elements}{%
\subsubsection{Off-diagonal elements}\label{off-diagonal-elements}}

\(\Omega_{ij}=\frac{1}{M}\sum_{k=1}^Ms_{ik}s_{jk}\)

and
\(\Omega_{ij}^2=\frac{1}{M^2}[\sum_{k=1}^Ms^2_{ik}s^2_{jk}+\sum_{k_1=1}^M\sum_{k_2=1}^Ms_{ik_1}s_{ik_2}s_{jk_1}s_{jk_2}]\)

in which \(E(s_{ik}^2s_{jk}^2)=E(s_{ik}^2)E(s_{jk}^2)\), if the
individuals are not related.

\(E(s_{ik_1}s_{ik_2}s_{jk_1}s_{jk_2})=E(s_{ik_1}s_{ik_2})E(s_{jk_1}s_{jk_2})=E(\rho_{i,k_1k_2})E(\rho_{j,k_1k_2})=E^2(\rho_{k_1k_2})\).

So

\[E(\mathbf{\Omega}_{ij}^2)=\frac{1}{M}+\frac{M-1}{M}\color{green}{E^2(\rho_{k_1k_2})}\]

\hypertarget{es2_iikes4_iik-vars_ii2}{%
\subsubsection{\texorpdfstring{\(E(s^2_{ii,k}),E(s^4_{ii,k}), var(s_{ii}^2)\)}{E(s\^{}2\_\{ii,k\}),E(s\^{}4\_\{ii,k\}), var(s\_\{ii\}\^{}2)}}\label{es2_iikes4_iik-vars_ii2}}

\hypertarget{some-useful-identities-for-s.}{%
\paragraph{\texorpdfstring{Some useful identities for
\(s\).}{Some useful identities for s.}}\label{some-useful-identities-for-s.}}

\[E(s^2_{ii,k})=1\]

\[E(\color{red} {s^4_{ii,k}})=\frac{1}{2p_kq_k}\] The
\(\color{red} {s^4_{ik} = \frac{1}{2p_kq_k}}\), in which \(2p_kq_k\) is
the expected value of the variance of the \(k^{th}\) locus under random
mating. \(\color{red} {s^4_{ik}}\) can be derived as below,

\begin{longtable}[]{@{}cccc@{}}
\toprule
\begin{minipage}[b]{0.20\columnwidth}\centering
\strut
\end{minipage} & \begin{minipage}[b]{0.20\columnwidth}\centering
\(a_ka_k\)\strut
\end{minipage} & \begin{minipage}[b]{0.24\columnwidth}\centering
\(A_ka_k\)\strut
\end{minipage} & \begin{minipage}[b]{0.24\columnwidth}\centering
\(A_kA_k\)\strut
\end{minipage}\tabularnewline
\midrule
\endhead
\begin{minipage}[t]{0.20\columnwidth}\centering
\strut
\end{minipage} & \begin{minipage}[t]{0.20\columnwidth}\centering
\(q_k^2\)\strut
\end{minipage} & \begin{minipage}[t]{0.24\columnwidth}\centering
\(2p_kq_k\)\strut
\end{minipage} & \begin{minipage}[t]{0.24\columnwidth}\centering
\(p_k^2\)\strut
\end{minipage}\tabularnewline
\begin{minipage}[t]{0.20\columnwidth}\centering
\(\Omega_{ii}=s^2_{ii}\)\strut
\end{minipage} & \begin{minipage}[t]{0.20\columnwidth}\centering
\(\frac{4p_k^2}{2p_kq_k}\)\strut
\end{minipage} & \begin{minipage}[t]{0.24\columnwidth}\centering
\(\frac{(1-2p_k)^2}{2p_kq_k}\)\strut
\end{minipage} & \begin{minipage}[t]{0.24\columnwidth}\centering
\(\frac{4q_k^2}{2p_kq_k}\)\strut
\end{minipage}\tabularnewline
\begin{minipage}[t]{0.20\columnwidth}\centering
\(\Omega_{ii}^2=s^4_{ii}\)\strut
\end{minipage} & \begin{minipage}[t]{0.20\columnwidth}\centering
\(\frac{16p_k^4}{4p_k^2q_k^2}\)\strut
\end{minipage} & \begin{minipage}[t]{0.24\columnwidth}\centering
\(\frac{(q_k-p_k)^4}{4p_k^2q_k^2}\)\strut
\end{minipage} & \begin{minipage}[t]{0.24\columnwidth}\centering
\(\frac{16p_k^4}{4p_k^2q_k^2}\)\strut
\end{minipage}\tabularnewline
\bottomrule
\end{longtable}

\[var(s_{ii}^2)=E(s_{ii}^4)-E^2(s_{ii}^2)=\frac{1}{2p_kq_k}-1\]

\begin{Shaded}
\begin{Highlighting}[]
\KeywordTok{library}\NormalTok{(Rcpp)}
\KeywordTok{sourceCpp}\NormalTok{(}\StringTok{"~/git/Notes/R/RLib/Shotgun.cpp"}\NormalTok{)}
\KeywordTok{source}\NormalTok{(}\StringTok{"~/git/Notes/R/RLib/shotgun.R"}\NormalTok{)}
\NormalTok{M=}\DecValTok{1000}
\NormalTok{N=}\DecValTok{500}
\NormalTok{frq=}\KeywordTok{runif}\NormalTok{(M, }\FloatTok{0.1}\NormalTok{, }\FloatTok{0.5}\NormalTok{)}
\NormalTok{Dp=}\KeywordTok{sample}\NormalTok{(}\KeywordTok{c}\NormalTok{(}\KeywordTok{runif}\NormalTok{(M}\OperatorTok{/}\DecValTok{2}\NormalTok{, }\DecValTok{0}\NormalTok{, }\FloatTok{.5}\NormalTok{), }\KeywordTok{runif}\NormalTok{(M}\OperatorTok{/}\DecValTok{2}\NormalTok{, }\FloatTok{-0.5}\NormalTok{, }\DecValTok{0}\NormalTok{)), M)}
\NormalTok{Dp=Dp[}\OperatorTok{-}\DecValTok{1}\NormalTok{]}

\NormalTok{G=}\KeywordTok{GenerateGenoDprimeRcpp}\NormalTok{(frq, Dp, N)}
\NormalTok{s=}\KeywordTok{apply}\NormalTok{(G, }\DecValTok{2}\NormalTok{, scale)}
\CommentTok{#Kcpp=CorMatrixRcpp(s)}
\NormalTok{FQ=}\KeywordTok{colMeans}\NormalTok{(G)}\OperatorTok{/}\DecValTok{2}
\NormalTok{s2=s}\OperatorTok{^}\DecValTok{2}
\NormalTok{s4=s}\OperatorTok{^}\DecValTok{4}
\KeywordTok{layout}\NormalTok{(}\KeywordTok{matrix}\NormalTok{(}\DecValTok{1}\OperatorTok{:}\DecValTok{3}\NormalTok{, }\DecValTok{1}\NormalTok{, }\DecValTok{3}\NormalTok{))}
\KeywordTok{hist}\NormalTok{(}\DataTypeTok{main=}\StringTok{"Simulation test for s^2"}\NormalTok{, }\KeywordTok{colMeans}\NormalTok{(s2), }\DataTypeTok{xlab=}\StringTok{"E(s^2)"}\NormalTok{)}
\KeywordTok{plot}\NormalTok{(}\DataTypeTok{main=}\StringTok{"Simulation test for s^4"}\NormalTok{, }\DecValTok{1}\OperatorTok{/}\NormalTok{(}\DecValTok{2}\OperatorTok{*}\NormalTok{FQ}\OperatorTok{*}\NormalTok{(}\DecValTok{1}\OperatorTok{-}\NormalTok{FQ)), }\KeywordTok{colMeans}\NormalTok{(s4), }\DataTypeTok{pch=}\DecValTok{16}\NormalTok{, }\DataTypeTok{cex=}\FloatTok{0.5}\NormalTok{, }\DataTypeTok{xlab=}\StringTok{"1/(2pq)"}\NormalTok{, }\DataTypeTok{ylab=}\StringTok{"Simulated s^4"}\NormalTok{)}
\KeywordTok{abline}\NormalTok{(}\DataTypeTok{a=}\DecValTok{0}\NormalTok{, }\DataTypeTok{b=}\DecValTok{1}\NormalTok{, }\DataTypeTok{col=}\StringTok{"red"}\NormalTok{, }\DataTypeTok{lwd=}\DecValTok{2}\NormalTok{, }\DataTypeTok{lty=}\DecValTok{2}\NormalTok{)}

\NormalTok{vs2=}\KeywordTok{apply}\NormalTok{(s2, }\DecValTok{2}\NormalTok{, var)}
\KeywordTok{plot}\NormalTok{(}\DataTypeTok{main=}\StringTok{"Siulation test for var(s^2)"}\NormalTok{,}\DecValTok{1}\OperatorTok{/}\NormalTok{(}\DecValTok{2}\OperatorTok{*}\NormalTok{FQ}\OperatorTok{*}\NormalTok{(}\DecValTok{1}\OperatorTok{-}\NormalTok{FQ))}\OperatorTok{-}\DecValTok{1}\NormalTok{, vs2, }\DataTypeTok{pch=}\DecValTok{16}\NormalTok{, }\DataTypeTok{cex=}\FloatTok{0.5}\NormalTok{, }\DataTypeTok{xlab=}\StringTok{"1/(2pq)-1"}\NormalTok{, }\DataTypeTok{ylab=}\StringTok{"Simulated var(s^2)"}\NormalTok{)}
\KeywordTok{abline}\NormalTok{(}\DataTypeTok{a=}\DecValTok{0}\NormalTok{, }\DataTypeTok{b=}\DecValTok{1}\NormalTok{, }\DataTypeTok{col=}\StringTok{"red"}\NormalTok{, }\DataTypeTok{lwd=}\DecValTok{2}\NormalTok{, }\DataTypeTok{lty=}\DecValTok{2}\NormalTok{)}

\CommentTok{#K}
\NormalTok{K=}\DecValTok{1}\OperatorTok{/}\NormalTok{M }\OperatorTok{*}\StringTok{ }\NormalTok{s }\OperatorTok\StringTok{ }\KeywordTok{t}\NormalTok{(s)}
\end{Highlighting}
\end{Shaded}

\hypertarget{es_iik_12s_iik_22-covs_iik_12s_iik_22}{%
\subsection{\texorpdfstring{3
\(E(s_{ii,k_1}^2s_{ii,k_2}^2), cov(s_{ii,k_1}^2,s_{ii,k_2}^2)\)}{3 E(s\_\{ii,k\_1\}\^{}2s\_\{ii,k\_2\}\^{}2), cov(s\_\{ii,k\_1\}\^{}2,s\_\{ii,k\_2\}\^{}2)}}\label{es_iik_12s_iik_22-covs_iik_12s_iik_22}}

\begin{align}
E(\color{green}{s_{ii,k_1}^2s_{ii,k_2}^2})&= \color{purple}{q^2_{k_1}\frac{4p_{k_1}^2}{2p_{k_1}q_{k_1}} [\frac{4p_{k_2}^2r^2_{k_1k_2}+(q_{k_2}-p_{k_2})^2 2r_{k_1k_2}\bar{r}_{k_1k_2} + 4q_{k_2}^2\bar{r}_{k_1k_2}^2}{2p_{k_2}q_{k_2}}]}\\
&+2p_{k_1k_2}\frac{(q_{k_1}-p_{k_2})^2}{2p_{k_1k_2}}[\frac{4p_{k_2}^2r_{k_1k_2}\bar{R}_{k_1k_2}+(q_{k_2}-p_{k_2})^2 (\bar{r}_{k_1k_2}\bar{R}_{k_1k_2}+r_{k_1k_2}R_{k_1k_2}) + 4q_{k_2}^2\bar{r}_{k_1k_2}R_{k_1k_2}}{2p_{k_2}q_{k_2}}]\\
&+\color{purple}{p^2_{k_1}\frac{4q_{k_1}^2}{2p_{k_1}q_{k_1}}[\frac{4p_{k_2}^2\bar{R}^2_{k_1k_2}+(q_{k_2}-p_{k_2})^2 2R_{k_1k_2}\bar{R}_{k_1k_2} + 4q_{k_2}^2R_{k_1k_2}^2}{2p_{k_2}q_{k_2}}]}\\
&=\color{purple}{2p_{k_1}q_{k_1}[\frac{4p^2_{k_2}(r^2_{k_1k_2}+\bar{R}_{k_1k_2})+(q_{k_2}-p_{k_2})^2(2r_{k_1k_2}\bar{r}_{k_1k_2}+2R_{k_1k_2}\bar{R}_{k_1k_2})+4q^2_{k_2}(\bar{r}^2_{k_1k_2}+R^2_{k_1k_2})}{2p_{k_2}{k_2}}]}\\
&+(1-4p_{k_1}q_{k_1})\frac{4p^2_{k_2}r_{k_1k_2}\bar{R}_{k_1k_2}+(q_{k_2}-p_{k_2})^2(\bar{r}_{k_1}{k_2}\bar{R}_{k_1}{k_2}+r_{k_1}{k_2}R_{k_1}{k_2})}{2p_{k_2}q_{k_2}})\\
&=1+\frac{D_{k_1k_2}[(p_{k_1}-q_{k_1})(p_{k_2}-q_{k_2})]}{2p_{k_1}q_{k_1}p_{k_2}q_{k_2}}-\frac{D^2_{k_1k_2}}{p_{k_1}q_{k_1}p_{k_2}q_{k_2}}\\
&=1+\rho^2_{k_1k_2}+\frac{\rho_{k_1k_2}(p_{k_1}-q_{k_1})(p_{k_2}-q_{k_2})}{2\sqrt{p_{k_1}q_{k_1}p_{k_2}q_{k_2}}}
\end{align}

And consequently, we can derive \begin{align}
cov(s_{ii,k_1}^2,s_{ii,k_2}^2)&=E(s_{i,k_1}^2s_{i,k_2}^2)-E^2(s_{i,k_1})E^2(s_{i,k_2})\\
&=1+\rho^2_{k_1k_2}+\frac{\rho_{k_1k_2}(p_{k_1}-q_{k_1})(p_{k_2}-q_{k_2})}{2\sqrt{p_{k_1}q_{k_1}p_{k_2}q_{k_2}}}-1\\
&=\rho^2_{k_1k_2}+\rho_{k_1k_2}\frac{(p_{k_1}-q_{k_1})(p_{k_2}-q_{k_2})}{2\sqrt{p_{k_1}q_{k_1}p_{k_2}q_{k_2}}}
\end{align}

When \(k_1 = k_2\), it is obviously that
\(\rho_{k_1,k_2}=1, p_{k_1}=p_{k_2}\), and
\[cov(s_{ii,k_1}^2,s_{ii,k_2}^2)=1+1\times\frac{(p_{k}-q_{k})^2}{2p_{k}q_{k}}=1+\frac{1-4p_{k}q_{k}}{2p_{k}q_{k}}=\frac{1}{2p_{k}q_{k}}-1=var(s_{ii,k}^2)\]

\hypertarget{m_e_a}{%
\subsection{\texorpdfstring{4 \(M_{E_a}\)}{4 M\_\{E\_a\}}}\label{m_e_a}}

UKB cohorts test, ``/public/home/xuhm/gc5k/UKB\_cohort/cohort/grm/sub''

\begin{Shaded}
\begin{Highlighting}[]
\NormalTok{ukMe=}\KeywordTok{read.table}\NormalTok{(}\StringTok{"~/manuscript/Linkage_Association/h2ss/meMarker.txt"}\NormalTok{, }\DataTypeTok{as.is =}\NormalTok{ T)}
\NormalTok{me=ukMe[,}\OperatorTok{-}\DecValTok{1}\NormalTok{]}
\KeywordTok{rownames}\NormalTok{(me)=ukMe[,}\DecValTok{1}\NormalTok{]}
\KeywordTok{layout}\NormalTok{(}\KeywordTok{matrix}\NormalTok{(}\DecValTok{1}\OperatorTok{:}\DecValTok{3}\NormalTok{, }\DecValTok{3}\NormalTok{, }\DecValTok{1}\NormalTok{))}

\NormalTok{me1=me[, }\DecValTok{1}\OperatorTok{:}\DecValTok{10}\NormalTok{]}
\NormalTok{MeS1=}\KeywordTok{matrix}\NormalTok{(}\KeywordTok{c}\NormalTok{(}\KeywordTok{apply}\NormalTok{(me1, }\DecValTok{1}\NormalTok{, mean), }\KeywordTok{apply}\NormalTok{(me1, }\DecValTok{1}\NormalTok{, sd)), }\KeywordTok{nrow}\NormalTok{(me1), }\DecValTok{2}\NormalTok{, }\DataTypeTok{byrow =}\NormalTok{ F)}
\KeywordTok{barplot}\NormalTok{(}\KeywordTok{t}\NormalTok{(}\KeywordTok{as.matrix}\NormalTok{(me1)), }\DataTypeTok{beside =}\NormalTok{ T)}

\NormalTok{me2=me[,}\DecValTok{11}\OperatorTok{:}\DecValTok{20}\NormalTok{]}
\NormalTok{MeS2=}\KeywordTok{matrix}\NormalTok{(}\KeywordTok{c}\NormalTok{(}\KeywordTok{apply}\NormalTok{(me2, }\DecValTok{1}\NormalTok{, mean), }\KeywordTok{apply}\NormalTok{(me2, }\DecValTok{1}\NormalTok{, sd)), }\KeywordTok{nrow}\NormalTok{(me2), }\DecValTok{2}\NormalTok{, }\DataTypeTok{byrow =}\NormalTok{ F)}
\KeywordTok{barplot}\NormalTok{(}\KeywordTok{t}\NormalTok{(}\KeywordTok{as.matrix}\NormalTok{(me2)), }\DataTypeTok{beside =}\NormalTok{ T)}

\NormalTok{me3=me[,}\DecValTok{21}\OperatorTok{:}\DecValTok{30}\NormalTok{]}
\NormalTok{MeS3=}\KeywordTok{matrix}\NormalTok{(}\KeywordTok{c}\NormalTok{(}\KeywordTok{apply}\NormalTok{(me3, }\DecValTok{1}\NormalTok{, mean), }\KeywordTok{apply}\NormalTok{(me3, }\DecValTok{1}\NormalTok{, sd)), }\KeywordTok{nrow}\NormalTok{(me3), }\DecValTok{2}\NormalTok{, }\DataTypeTok{byrow =}\NormalTok{ F)}
\KeywordTok{barplot}\NormalTok{(}\KeywordTok{t}\NormalTok{(}\KeywordTok{as.matrix}\NormalTok{(me3)), }\DataTypeTok{beside =}\NormalTok{ T)}
\end{Highlighting}
\end{Shaded}

\includegraphics{H2ss_files/figure-latex/me-1.pdf}

\begin{Shaded}
\begin{Highlighting}[]
\NormalTok{MeMean=}\KeywordTok{matrix}\NormalTok{(}\DecValTok{0}\NormalTok{, }\DataTypeTok{nrow=}\KeywordTok{nrow}\NormalTok{(me), }\DecValTok{3}\NormalTok{)}
\NormalTok{MeMean[,}\DecValTok{1}\NormalTok{]=}\KeywordTok{apply}\NormalTok{(me1, }\DecValTok{1}\NormalTok{, mean)}
\NormalTok{MeMean[,}\DecValTok{2}\NormalTok{]=}\KeywordTok{apply}\NormalTok{(me2, }\DecValTok{1}\NormalTok{, mean)}
\NormalTok{MeMean[,}\DecValTok{3}\NormalTok{]=}\KeywordTok{apply}\NormalTok{(me3, }\DecValTok{1}\NormalTok{, mean)}

\NormalTok{MeSd=}\KeywordTok{matrix}\NormalTok{(}\DecValTok{0}\NormalTok{, }\DataTypeTok{nrow=}\KeywordTok{nrow}\NormalTok{(me), }\DecValTok{3}\NormalTok{)}
\NormalTok{MeSd[,}\DecValTok{1}\NormalTok{]=}\KeywordTok{apply}\NormalTok{(me1, }\DecValTok{1}\NormalTok{, sd)}
\NormalTok{MeSd[,}\DecValTok{2}\NormalTok{]=}\KeywordTok{apply}\NormalTok{(me2, }\DecValTok{1}\NormalTok{, sd)}
\NormalTok{MeSd[,}\DecValTok{3}\NormalTok{]=}\KeywordTok{apply}\NormalTok{(me3, }\DecValTok{1}\NormalTok{, sd)}

\NormalTok{MeSd=}\KeywordTok{matrix}\NormalTok{(}\DecValTok{0}\NormalTok{, }\DataTypeTok{nrow=}\KeywordTok{nrow}\NormalTok{(me), }\DecValTok{3}\NormalTok{)}
\NormalTok{MeSd[,}\DecValTok{1}\NormalTok{]=}\KeywordTok{apply}\NormalTok{(me1, }\DecValTok{1}\NormalTok{, sd)}
\NormalTok{MeSd[,}\DecValTok{2}\NormalTok{]=}\KeywordTok{apply}\NormalTok{(me2, }\DecValTok{1}\NormalTok{, sd)}
\NormalTok{MeSd[,}\DecValTok{3}\NormalTok{]=}\KeywordTok{apply}\NormalTok{(me3, }\DecValTok{1}\NormalTok{, sd)}

\KeywordTok{layout}\NormalTok{(}\KeywordTok{matrix}\NormalTok{(}\DecValTok{1}\NormalTok{,}\DecValTok{1}\NormalTok{,}\DecValTok{1}\NormalTok{))}
\NormalTok{barP=}\KeywordTok{barplot}\NormalTok{(MeMean, }\DataTypeTok{beside =}\NormalTok{ T, }\DataTypeTok{ylim=}\KeywordTok{c}\NormalTok{(}\DecValTok{0}\NormalTok{, }\DecValTok{60000}\NormalTok{))}
\KeywordTok{segments}\NormalTok{(barP, MeMean}\OperatorTok{-}\NormalTok{MeSd}\OperatorTok{*}\DecValTok{2}\NormalTok{, barP, MeMean}\OperatorTok{+}\NormalTok{MeSd}\OperatorTok{*}\DecValTok{2}\NormalTok{)}
\end{Highlighting}
\end{Shaded}

\includegraphics{H2ss_files/figure-latex/me-2.pdf}

\hypertarget{grm-mathbfomega_d}{%
\subsection{\texorpdfstring{GRM
\(\mathbf{\Omega}_d\)}{GRM \textbackslash{}mathbf\{\textbackslash{}Omega\}\_d}}\label{grm-mathbfomega_d}}

\begin{longtable}[]{@{}cccc@{}}
\caption{Coding for genotypes}\tabularnewline
\toprule
Genotype & \(aa\) & \(Aa\) & \(AA\)\tabularnewline
\midrule
\endfirsthead
\toprule
Genotype & \(aa\) & \(Aa\) & \(AA\)\tabularnewline
\midrule
\endhead
Additive \(x_A\) & 0 & 1 & 2\tabularnewline
Dominance \(x_d\) & 0 & \(2p\) & \(4p-2\)\tabularnewline
Frequency & \(q^2\) & \(2pq\) & \(p^2\)\tabularnewline
\bottomrule
\end{longtable}

We have \(E(x_A)=2p\), \(E(x_d)=2p^2\), and
\(E(x_ax_d)=4p^2q+8p^3-4p^2=4p^3\), and consequently,
\(cov(x_a, x_d)=E(x_ax_d)-E(x_a)E(x_d)=0\).

For a pair of individuals, their relatedness in terms of dominance
effect is \[
\Omega_{d_{ij}}=\frac{1}{M}\sum_{k=1}^M\frac{(x_{d_{i,k}}-2p_k^2)(x_{d_{j,k}}-2p_k^2)}{4p_k^2(1-p_k)^2}
\] The effective number of markers is
\(E(m_{e_d})=\frac{M^2}{\sum_{l_1=1}^M\sum_{l_2=1}^M\rho_{l_1l_2}^4}\).

\hypertarget{lse-phenome}{%
\subsection{LSE (Phenome)}\label{lse-phenome}}

For a linear model \(\mathbf{y}=\mathbf{X\beta}+e\),

We have the target function
\[Q=e^Te=argmin_{\mathbf{\beta}}(\mathbf{y}-\mathbf{X\beta})^T(\mathbf{y}-\mathbf{X\beta})\]
It can be minized with OLS as below \[\begin{align}
\frac{\partial Q}{\partial \beta}&=\frac{(\mathbf{y}-\mathbf{X\beta})^T(\mathbf{y}-\mathbf{X\beta})}{\partial \beta}\\
&=\frac{\partial [y^Ty-\color{red}{\mathbf{\beta}^T\mathbf{X}^Ty}-\color{green}{y^T\mathbf{X\beta}}+\color{blue}{\mathbf{\beta}^T\mathbf{X}^T\mathbf{X}\beta}]}{\partial \mathbf{\beta}}\\
&=0-\color{red}{\mathbf{Xy}^T}-\color{green}{\mathbf{X}\mathbf{y}^T}+\color{blue}{[\mathbf{X}^T\mathbf{X}+(\mathbf{X}^T\mathbf{X})^T]\mathbf{\beta}}, (side\ notes, \mathbf{X}^T\mathbf{X}=(\mathbf{X}^T\mathbf{X})^T)\\
&=-2\mathbf{Xy}^T-2\mathbf{X^TX}\beta
\end{align}
\] Then, we have

\[\hat{\mathbf{\beta}}=(\mathbf{X}^T\mathbf{X})^{-1}\mathbf{X}^T\mathbf{y}\]

\hypertarget{notes-for-matrix-calculus}{%
\paragraph{Notes for matrix calculus}\label{notes-for-matrix-calculus}}

\textbf{1:} more topics about matrix calculus please refer to
\href{https://en.wikipedia.org/wiki/Matrix_calculus\#Derivatives_with_vectors}{``Matrix
Calculus Wiki''}.

\textbf{2:} Or read a gentle introduction
\href{https://arxiv.org/abs/1802.01528}{``Matrix Calculus You Need for
Deep Learning''}.

\hypertarget{glse}{%
\paragraph{GLSE}\label{glse}}

If it is general linear model, the LSE is to \emph{minimize}
\(Q=argmin_{\mathbf{\beta}}{(\mathbf{y}-\mathbf{X\beta})^TV^{-1}(\mathbf{y}-\mathbf{X\beta})}\),
and its corresponding
\(\hat{\mathbf{\beta}}=(\mathbf{X}^T\mathbf{V}^{-1}\mathbf{X})^{-1}(\mathbf{X}^T\mathbf{V}^{-1}\mathbf{y})\).

\hypertarget{the-sampling-variance-of-hatmathbfbeta-is}{%
\paragraph{\texorpdfstring{The sampling variance of
\(\hat{\mathbf{\beta}}\)
is}{The sampling variance of \textbackslash{}hat\{\textbackslash{}mathbf\{\textbackslash{}beta\}\} is}}\label{the-sampling-variance-of-hatmathbfbeta-is}}

\begin{align}
SSE&=\color{red}{e^T}\color{green}{e}\\
&=\color{red}{(\mathbf{y}-\mathbf{X\hat{\beta}})^T}\color{green}{(\mathbf{y}-\mathbf{X\hat{\beta}})}\\
&=\color{red}{[\mathbf{y}-\mathbf{X}(\mathbf{X}^T\mathbf{X})^{-1}\mathbf{X}^{T}\mathbf{y}]^T}\color{green}{[\mathbf{y}-\mathbf{X}(\mathbf{X}^T\mathbf{X})^{-1}\mathbf{X}^T\mathbf{y}]}\\
&=\color{red}{\mathbf{y}^T}\color{green}{\mathbf{y}}-\color{red}{\mathbf{y}^T}\color{green}{\mathbf{X}(\mathbf{X}^T\mathbf{X})^{-1}\mathbf{X}^T\mathbf{y}}-\color{red}{\mathbf{y}{\mathbf{X}(\mathbf{X}^T\mathbf{X})^{-1}\mathbf{X}^T}}\color{green}{\mathbf{y}}+\color{red}{\mathbf{y}^T\mathbf{X}(\mathbf{X}^T\mathbf{X})^{-1}\mathbf{X}^{T}}\color{green}{\mathbf{X}(\mathbf{X}^T\mathbf{X})^{-1}\mathbf{X}^{T}\mathbf{y}}\\
&=\mathbf{y}^T\mathbf{y}-\mathbf{y}^T\mathbf{X}(\mathbf{X}^T\mathbf{X})^{-1}\mathbf{X}^T\mathbf{y}\\
&=\mathbf{y}^T(I-\mathbf{P}_{\mathbf{X}})\mathbf{y}
\end{align}

Notes:
\(\mathbf{P}_{\mathbf{X}}=\mathbf{X}(\mathbf{X}^T\mathbf{X})^{-1}\mathbf{X}^T\)
is
\href{https://en.wikipedia.org/wiki/Projection_(linear_algebra)}{projection
matrix wiki}, that \(\mathbf{P}^n=\mathbf{P}\).

\[\hat{\sigma}_{\beta}^2=\frac{SSE}{n-m}(\mathbf{X}^T\mathbf{X})^{-1}=\hat{\sigma}^2_e(\mathbf{X}^T\mathbf{X})^{-1}\]

\hypertarget{variation-for-lse}{%
\subsubsection{Variation for LSE}\label{variation-for-lse}}

\hypertarget{slr-and-mlr}{%
\paragraph{SLR and MLR}\label{slr-and-mlr}}

There is connection of the regression coefficients between simple linear
regresssion and multiple linear regrression.

\[\begin{align}
\hat{\mathbf{\beta}}&=&(\mathbf{X}^T\mathbf{X})^{-1}\mathbf{X}^T\mathbf{y}\\
&=&(\Lambda_{\sigma_X}\Sigma\Lambda_{\sigma_X})^{-1}\rho_{Xy}\Lambda_{\sigma_X}\sigma_y\\
&=&\Lambda_{\sigma_X}^{-1}\Sigma^{-1}\Lambda_{\sigma_X}^{-1}\rho_{Xy}\Lambda_{\sigma_X}\sigma_y\\
&=&\Lambda_{\sigma_X}^{-1}\Sigma^{-1}\rho_{Xy}\sigma_y\\
&=&\Sigma^{-1}\hat{b}_{Xy}\\
\end{align}
\]

in which
\(\hat{b}_{[i]}=\frac{cov(y, x_i)}{var(x_i)}=\rho_{y,x_i}\frac{\sigma_y}{\sigma_{x_i}}\),
estimated from the simple linear regression from \(y=a+b_ix_i+e\).

And similarly,
\(\hat{\sigma}^2_{\beta}=(\frac{y^Ty-\hat{\mathbf{\beta}}^T\hat{b}}{n-m})(\mathbf{\Lambda_{\sigma_X}\Sigma\Lambda_{\sigma_X}})^{-1}\)

\includegraphics{H2ss_files/figure-latex/g3-1.pdf}

\hypertarget{real-example}{%
\paragraph{Real example}\label{real-example}}

`/public/home/xuhm/gc5k/UKB\_cohort/ss\_adj'

\begin{longtable}[]{@{}cccc@{}}
\toprule
cohort & \(gender\) & \(ht\) & \(cor\)\tabularnewline
\midrule
\endhead
Barts & 1 & 1 & -0.6918\tabularnewline
Birmingham & 1 & 1 & -0.7139\tabularnewline
Bristol & 1 & 1 & -0.7183\tabularnewline
Bury & 1 & 1 & -0.7131\tabularnewline
Cardiff & 1 & 1 & -0.7244\tabularnewline
Croydon & 1 & 1 & -0.7133\tabularnewline
Edinburgh & 1 & 1 & -0.7114\tabularnewline
Glasgow & 1 & 1 & -0.7160\tabularnewline
Hounslow & 1 & 1 & -0.7170\tabularnewline
Leeds & 1 & 1 & -0.7136\tabularnewline
Liverpool & 1 & 1 & -0.7170\tabularnewline
Manchester & 1 & 1 & -0.6995\tabularnewline
Middlesborough & 1 & 1 & -0.7199\tabularnewline
Newcastle & 1 & 1 & -0.7173\tabularnewline
Nottingham & 1 & 1 & -0.7177\tabularnewline
Oxford & 1 & 1 & -0.7153\tabularnewline
Reading & 1 & 1 & -0.7252\tabularnewline
Sheffield & 1 & 1 & -0.7180\tabularnewline
Stoke & 1 & 1 & -0.7164\tabularnewline
\bottomrule
\end{longtable}

\hypertarget{quadratic-form-statistics}{%
\subsection{Quadratic form statistics}\label{quadratic-form-statistics}}

Let \(x\) be a vector for \(n\) random variables, and \(A\) be a
\(n\times n\) symmetric matrix, the scalar quantity \(xAx\) is known as
quadratic form. If \[x \sim MVN(\mu,\sum)\], it can be shown that

\[E[xAx]=tr[A\sum]+\mu^TA\mu\]

It can be proved as below \(x^TAx = tr[x^TAx]\), and by the cyclic
property of the trace operator \(E[tr(x^TAx)]=E[tr[Axx^T]]\). Since
trace operator is linear combination of matrices, it therefor follows
from the linearity of the expectation operator that
\[E[tr(Axx^T)]=tr[AE(xx^T)]=tr[A(\sum+\mu\mu^T)]=tr(A\sum)+tr(\sum\mu\mu^T)=tr(A\sum)+tr(\mu^T\sum\mu)=tr(A\sum)+\mu^TA\mu\].

\begin{longtable}[]{@{}l@{}}
\toprule
\endhead
\begin{minipage}[t]{0.11\columnwidth}\raggedright
Variance \[var[\mu^TA\mu]=2tr[A\sum A\sum]+4\mu^TA\sum A\mu\] Covariance
\[cov[\mu^TA_1, \mu A_2]=2tr[A_1\sum A_2\sum]+4\mu^TA_1\sum A_2\mu\] for
both variance and covariance, \(A\) is required to be
\textbf{symmetric}.\strut
\end{minipage}\tabularnewline
\begin{minipage}[t]{0.11\columnwidth}\raggedright
When \(A\) is not \textbf{symmetric}\strut
\end{minipage}\tabularnewline
\begin{minipage}[t]{0.11\columnwidth}\raggedright
When \(A\) is asymmetric, \[x^T\tilde{A}x=x^T(A+A^T)x/2\] the expression
of var and cov are the same, provided \(A\) is replaced by
\(\tilde{A}=(A+A^T)/2\).\strut
\end{minipage}\tabularnewline
\begin{minipage}[t]{0.11\columnwidth}\raggedright
\#\# Isserlis theorem It may be related to high-order moments for
multivariate normal distribution, see
\href{https://en.wikipedia.org/wiki/Isserlis\%27_theorem}{Isserlis
Theorem}, published in \emph{Biometrika},
12:\href{https://www.jstor.org/stable/2331932?seq=1}{134-139}.\strut
\end{minipage}\tabularnewline
\begin{minipage}[t]{0.11\columnwidth}\raggedright
\(X_i\) is a vector that follows standard Gaussian distribution
\(N(0,1)\). The product below can be expanded as\strut
\end{minipage}\tabularnewline
\begin{minipage}[t]{0.11\columnwidth}\raggedright
\#\#\#\# Example 1 (4-order moments)\strut
\end{minipage}\tabularnewline
\begin{minipage}[t]{0.11\columnwidth}\raggedright
\begin{align}
E(X_1X_2X_3X_4)&=E(X_1X_2)E(X_3X_4)+E(X_1X_3)E(X_2X_4)+E(X_1X_4)E(X_2X_3)
\end{align} When \(X_i\), \(i=1\), it is \(E(X_1^4)=3E(X_1^2)=3\).\strut
\end{minipage}\tabularnewline
\begin{minipage}[t]{0.11\columnwidth}\raggedright
\#\#\#\# Example 2 (6-order moments) \begin{align}
E(X_1X_2X_3X_4X_5X_5)&=E(X_1X_2)E(X_3X_4X_5X_6)+E(X_1X_3)E(X_2X_4X_5X_6)+E(X_1X_4)E(X_2X_3X_5X_6)+E(X_1X_5)E(X_2X_3X_4X_6)+E(X_1X_6)E(X_2X_3X_4X_5), (5\ terms)\\
&=E(X_1X_2)[E(X_3X_4)E(X_5X_6)+E(X_3X_5)E(X_4X_6)+E(X_3X_6)E(X_4X_5)]+...,(E(X_3X_4)E(X_5X_6)\ can\ be\ expanded\ as\ Example\ 1)\\
&=15
\end{align}\strut
\end{minipage}\tabularnewline
\begin{minipage}[t]{0.11\columnwidth}\raggedright
However, Isserlis Theorem works for normal distribution only, but it is
not ``normal'' for a standardized locus.\strut
\end{minipage}\tabularnewline
\begin{minipage}[t]{0.11\columnwidth}\raggedright
\#\#\#\# Cumulant Another possible method is ``Cumulant'', see its
\href{https://en.wikipedia.org/wiki/Cumulant}{Wiki}. Or for ``joint
Cumulant'' if covariance exists beween two vectors, see
\href{https://en.wikipedia.org/wiki/Law_of_total_cumulance}{Joint
Cumulant}.\strut
\end{minipage}\tabularnewline
\begin{minipage}[t]{0.11\columnwidth}\raggedright
\#\# Inversion\strut
\end{minipage}\tabularnewline
\begin{minipage}[t]{0.11\columnwidth}\raggedright
\#\#\# \(2 \times 2\) \[\mathbf{A}^{-1}=
\begin{bmatrix} a & b \\ c & d \end{bmatrix} ^{-1} = \frac{1}{ad-bc}\begin{bmatrix} d & a \\ -b & -c \end{bmatrix}\]\strut
\end{minipage}\tabularnewline
\begin{minipage}[t]{0.11\columnwidth}\raggedright
\#\#\# \(3 \times 3\) \[\mathbf{A}^{-1}=
\begin{bmatrix} a & b &c \\ d & e & f \\g&h&i \end{bmatrix} ^{-1} = \frac{1}{det(\mathbf{A})}\begin{bmatrix} A & D &G \\B&E&H \\C & F &I \end{bmatrix}\]
in which (\(A\) is scalar not a matrix as above) \(A=(ei-fh)\),
\(B=-(di-fg)\), \(C=(dh-eg)\), \(D=-(bi-ch)\), \(E=(ai-cg)\),
\(F=-(ah-bg)\), \(G=(bf-ce)\), \(H=-(af-cd)\), \(I=(ae-bd)\).\strut
\end{minipage}\tabularnewline
\begin{minipage}[t]{0.11\columnwidth}\raggedright
\#\# Conditional probability\strut
\end{minipage}\tabularnewline
\begin{minipage}[t]{0.11\columnwidth}\raggedright
\#\#\# Haplotype for a pair of biallelic loci\strut
\end{minipage}\tabularnewline
\bottomrule
\end{longtable}

\begin{longtable}[]{@{}cccc@{}}
\toprule
\begin{minipage}[b]{0.22\columnwidth}\centering
\strut
\end{minipage} & \begin{minipage}[b]{0.22\columnwidth}\centering
\(a_l\)\strut
\end{minipage} & \begin{minipage}[b]{0.22\columnwidth}\centering
\(A_l\)\strut
\end{minipage} & \begin{minipage}[b]{0.22\columnwidth}\centering
\strut
\end{minipage}\tabularnewline
\midrule
\endhead
\begin{minipage}[t]{0.22\columnwidth}\centering
\(a_k\)\strut
\end{minipage} & \begin{minipage}[t]{0.22\columnwidth}\centering
\(r_{kl}=q_l+\frac{D_{kl}}{q_k}\)\strut
\end{minipage} & \begin{minipage}[t]{0.22\columnwidth}\centering
\(\bar{r}_{kl}=q_l+\frac{D_{kl}}{q_k}\)\strut
\end{minipage} & \begin{minipage}[t]{0.22\columnwidth}\centering
\(q_k\)\strut
\end{minipage}\tabularnewline
\begin{minipage}[t]{0.22\columnwidth}\centering
\(A_k\)\strut
\end{minipage} & \begin{minipage}[t]{0.22\columnwidth}\centering
\(\bar{R}_{kl}=q_l+\frac{D_{kl}}{p_{kl}}\)\strut
\end{minipage} & \begin{minipage}[t]{0.22\columnwidth}\centering
\(R_{kl}=p_l+\frac{D_{kl}}{p_{kl}}\)\strut
\end{minipage} & \begin{minipage}[t]{0.22\columnwidth}\centering
\(p_k\)\strut
\end{minipage}\tabularnewline
\begin{minipage}[t]{0.22\columnwidth}\centering
\strut
\end{minipage} & \begin{minipage}[t]{0.22\columnwidth}\centering
\(q_l\)\strut
\end{minipage} & \begin{minipage}[t]{0.22\columnwidth}\centering
\(p_l\)\strut
\end{minipage} & \begin{minipage}[t]{0.22\columnwidth}\centering
\strut
\end{minipage}\tabularnewline
\bottomrule
\end{longtable}

\begin{center}\rule{0.5\linewidth}{0.5pt}\end{center}

\(p_k\) and \(p_l\) are minor allelic frequencies of the two loci,
respectively. It exists the transformation as
\(D_{kl}=\rho_{kl}\sqrt{p_kq_kp_lq_l}\). Of note, \(D_{kl}\) is
restrained as \texttt{min}(\(p_kq_l\), \(q_kp_l\)), and consequently
\(\rho_{kl}=\sqrt{\frac{p_kq_l}{p_lq_k}}<1\).

\hypertarget{frequencies-for-a-pair-of-loci-for-the-ith-individual}{%
\subsubsection{\texorpdfstring{\textbf{Frequencies for a pair of loci
for the \(i^{th}\)
individual}}{Frequencies for a pair of loci for the i\^{}\{th\} individual}}\label{frequencies-for-a-pair-of-loci-for-the-ith-individual}}

\begin{center}\rule{0.5\linewidth}{0.5pt}\end{center}

\begin{longtable}[]{@{}ccccccc@{}}
\toprule
\begin{minipage}[b]{0.12\columnwidth}\centering
\strut
\end{minipage} & \begin{minipage}[b]{0.12\columnwidth}\centering
\strut
\end{minipage} & \begin{minipage}[b]{0.12\columnwidth}\centering
\strut
\end{minipage} & \begin{minipage}[b]{0.12\columnwidth}\centering
\strut
\end{minipage} & \begin{minipage}[b]{0.12\columnwidth}\centering
\strut
\end{minipage} & \begin{minipage}[b]{0.12\columnwidth}\centering
Locus \(l\)\strut
\end{minipage} & \begin{minipage}[b]{0.12\columnwidth}\centering
\strut
\end{minipage}\tabularnewline
\midrule
\endhead
\begin{minipage}[t]{0.12\columnwidth}\centering
\strut
\end{minipage} & \begin{minipage}[t]{0.12\columnwidth}\centering
Geno\strut
\end{minipage} & \begin{minipage}[t]{0.12\columnwidth}\centering
\strut
\end{minipage} & \begin{minipage}[t]{0.12\columnwidth}\centering
\strut
\end{minipage} & \begin{minipage}[t]{0.12\columnwidth}\centering
\(a_la_l\)\strut
\end{minipage} & \begin{minipage}[t]{0.12\columnwidth}\centering
\(a_lA_l\)\strut
\end{minipage} & \begin{minipage}[t]{0.12\columnwidth}\centering
\(A_lA_l\)\strut
\end{minipage}\tabularnewline
\begin{minipage}[t]{0.12\columnwidth}\centering
\strut
\end{minipage} & \begin{minipage}[t]{0.12\columnwidth}\centering
\strut
\end{minipage} & \begin{minipage}[t]{0.12\columnwidth}\centering
Standardized geno\strut
\end{minipage} & \begin{minipage}[t]{0.12\columnwidth}\centering
\strut
\end{minipage} & \begin{minipage}[t]{0.12\columnwidth}\centering
\(s_{il}=\frac{-2p_l}{\sqrt{2p_lq_l}}\)\strut
\end{minipage} & \begin{minipage}[t]{0.12\columnwidth}\centering
\(s_{il}=\frac{q_l-p_l}{\sqrt{2p_lq_l}}\)\strut
\end{minipage} & \begin{minipage}[t]{0.12\columnwidth}\centering
\(s_{il}=\frac{2q_l}{\sqrt{2p_lq_l}}\)\strut
\end{minipage}\tabularnewline
\begin{minipage}[t]{0.12\columnwidth}\centering
\strut
\end{minipage} & \begin{minipage}[t]{0.12\columnwidth}\centering
\strut
\end{minipage} & \begin{minipage}[t]{0.12\columnwidth}\centering
\strut
\end{minipage} & \begin{minipage}[t]{0.12\columnwidth}\centering
Frequency\strut
\end{minipage} & \begin{minipage}[t]{0.12\columnwidth}\centering
\(q_l^2\)\strut
\end{minipage} & \begin{minipage}[t]{0.12\columnwidth}\centering
\(2p_lq_l\)\strut
\end{minipage} & \begin{minipage}[t]{0.12\columnwidth}\centering
\(p_l^2\)\strut
\end{minipage}\tabularnewline
\begin{minipage}[t]{0.12\columnwidth}\centering
\strut
\end{minipage} & \begin{minipage}[t]{0.12\columnwidth}\centering
\(a_ka_k\)\strut
\end{minipage} & \begin{minipage}[t]{0.12\columnwidth}\centering
\(s_{ik}=\frac{-2p_k}{\sqrt{2p_kq_k}}\)\strut
\end{minipage} & \begin{minipage}[t]{0.12\columnwidth}\centering
\(q_k^2\)\strut
\end{minipage} & \begin{minipage}[t]{0.12\columnwidth}\centering
\(r_{kl}^2\)\strut
\end{minipage} & \begin{minipage}[t]{0.12\columnwidth}\centering
\(2r_{kl}\bar{r}_{kl}\)\strut
\end{minipage} & \begin{minipage}[t]{0.12\columnwidth}\centering
\(\bar{r}_{kl}^2\)\strut
\end{minipage}\tabularnewline
\begin{minipage}[t]{0.12\columnwidth}\centering
Locus \(k\)\strut
\end{minipage} & \begin{minipage}[t]{0.12\columnwidth}\centering
\(a_ka_k\)\strut
\end{minipage} & \begin{minipage}[t]{0.12\columnwidth}\centering
\(s_{ik}=\frac{q_k-p_k}{\sqrt{2p_kq_k}}\)\strut
\end{minipage} & \begin{minipage}[t]{0.12\columnwidth}\centering
\(q_k^2\)\strut
\end{minipage} & \begin{minipage}[t]{0.12\columnwidth}\centering
\(r_{kl}\bar{R}_{kl}\)\strut
\end{minipage} & \begin{minipage}[t]{0.12\columnwidth}\centering
\(r_{kl}\bar{R}_{kl}+\bar{r}_{kl}R_{kl}\)\strut
\end{minipage} & \begin{minipage}[t]{0.12\columnwidth}\centering
\(\bar{r}_{kl}R_{kl}\)\strut
\end{minipage}\tabularnewline
\begin{minipage}[t]{0.12\columnwidth}\centering
\strut
\end{minipage} & \begin{minipage}[t]{0.12\columnwidth}\centering
\(A_kA_k\)\strut
\end{minipage} & \begin{minipage}[t]{0.12\columnwidth}\centering
\(s_{ik}=\frac{2q_k}{\sqrt{2p_kq_k}}\)\strut
\end{minipage} & \begin{minipage}[t]{0.12\columnwidth}\centering
\(p_k^2\)\strut
\end{minipage} & \begin{minipage}[t]{0.12\columnwidth}\centering
\(\bar{R}_{kl}^2\)\strut
\end{minipage} & \begin{minipage}[t]{0.12\columnwidth}\centering
\(2R_{kl}\bar{R}_{kl}\)\strut
\end{minipage} & \begin{minipage}[t]{0.12\columnwidth}\centering
\(R_{kl}^2\)\strut
\end{minipage}\tabularnewline
\bottomrule
\end{longtable}

\begin{center}\rule{0.5\linewidth}{0.5pt}\end{center}

\hypertarget{r-sess}{%
\subsection{R sess}\label{r-sess}}

\begin{Shaded}
\begin{Highlighting}[]
\KeywordTok{sessionInfo}\NormalTok{()}
\end{Highlighting}
\end{Shaded}

\begin{verbatim}
## R version 3.6.2 (2019-12-12)
## Platform: x86_64-apple-darwin15.6.0 (64-bit)
## Running under: macOS Catalina 10.15.4
## 
## Matrix products: default
## BLAS:   /Library/Frameworks/R.framework/Versions/3.6/Resources/lib/libRblas.0.dylib
## LAPACK: /Library/Frameworks/R.framework/Versions/3.6/Resources/lib/libRlapack.dylib
## 
## locale:
## [1] en_US.UTF-8/en_US.UTF-8/en_US.UTF-8/C/en_US.UTF-8/en_US.UTF-8
## 
## attached base packages:
## [1] stats     graphics  grDevices utils     datasets  methods   base     
## 
## other attached packages:
## [1] kableExtra_1.1.0 knitr_1.28       Rcpp_1.0.3      
## 
## loaded via a namespace (and not attached):
##  [1] rstudioapi_0.11   xml2_1.2.2        magrittr_1.5      hms_0.5.3        
##  [5] munsell_0.5.0     rvest_0.3.5       viridisLite_0.3.0 colorspace_1.4-1 
##  [9] R6_2.4.1          rlang_0.4.5       stringr_1.4.0     httr_1.4.1       
## [13] tools_3.6.2       webshot_0.5.2     xfun_0.12         htmltools_0.4.0  
## [17] yaml_2.2.1        digest_0.6.25     lifecycle_0.2.0   tibble_2.1.3     
## [21] crayon_1.3.4      readr_1.3.1       vctrs_0.2.3       glue_1.3.1       
## [25] evaluate_0.14     rmarkdown_2.1     stringi_1.4.6     compiler_3.6.2   
## [29] pillar_1.4.3      scales_1.1.0      pkgconfig_2.0.3
\end{verbatim}

\end{document}
